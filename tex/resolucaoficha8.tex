% !TeX program = lualatex
\documentclass[a4paper]{article}
\usepackage{amsmath}
\usepackage[utf8]{inputenc}
\usepackage{amsmath}
\usepackage{cancel}
\usepackage[portuguese]{babel}
\usepackage{fancybox}
\usepackage{amssymb}
\usepackage{capt-of}
\usepackage{pgfplots}
\usepackage{tikz}
\usepackage{polynom}
\usepackage{adjustbox}
\usepackage{pgfplots}
\usepgfplotslibrary{fillbetween}
\usetikzlibrary{matrix}
\usetikzlibrary{calc}
\usetikzlibrary{patterns}
\usetikzlibrary{decorations.pathreplacing}

\tikzset{
	CE/.style={column #1/.style={nodes={text width=29mm}}}
}
\tikzset{
	CA/.style={column #1/.style={nodes={text width=35mm}}}
}

\usepackage{luacode}
\begin{luacode}
	-- Computes the roots of x^2+bx+c=0
	-- Returns nothing if they aren't real
	function roots(b, c)
	local delta = b*b-4*c
	if delta > 0 then
	deltasq = math.sqrt(delta)
	local r1 = math.round((-b-deltasq)/2)
	local r2 = math.round((-b+deltasq)/2)
	return r1, r2
	end
	end
	
	-- Outputs x^2+bx+c in developed form to LaTeX
	function display_polynome(b, c)
	p = "x^2"
	if b > 0 then
	p = p .. "+" .. tostring(b) .. "x"
	elseif b < 0 then
	p = p .. tostring(b) .. "x"
	end
	if c > 0 then
	p = p .. "+" .. tostring(c)
	elseif c < 0 then
	p = p .. tostring(c)
	end
	tex.print(p)
	end
	
	-- Outputs x^2+bx+c in factorized form to LaTeX
	function display_factorized_polynom(b, c) 
	r1, r2 = roots(b, c)
	p = "(x"
	if r1 > 0 then
	p = p .. "-" .. tostring(r1) .. ")(x"
	elseif r1 < 0 then
	p = p .. "+" .. tostring(-r1) .. ")(x"
	end
	if r2 > 0 then
	p = p .. "-" .. tostring(r2) .. ")"
	elseif r2 < 0 then
	p = p .. "+" .. tostring(-r2) .. ")"
	end
	tex.print(p)
	end
\end{luacode}

\begin{document}
	
	\section*{Exercício 1}\textbf{Considere a função quadrática definida por $f(x) = x^2 - 2x - 3$, $x \in \mathbb{R}$.}
\subsection*{a)}
\textbf{Escreva a expressão $x^2 -2x - 3$ na forma $\left(x- h\right)^2 + k$.}
\[h=-\frac{b}{2a}=\frac{2}{2}=1\]\
\[k=-\frac{\Delta}{4a}\]
\[\Delta=16, \text{então},  k=-\frac{16}{4}=-4\]
\text{Então a expressão $x^2 -2x - 3$ é equivalente a $\left(x - 1\right)^2 - 4$.}
\subsection*{b)}
\textbf{Calcule os zeros de $\it{f}$.}
\[x=\frac{2\pm\sqrt{16}}{2}\]
\[\Leftrightarrow x=\frac{2\pm4}{2}\]
\[\Leftrightarrow x=3 \lor x=-1\]
\text{Os zeros de $x^2 -2x - 3$ são $-1,3.	$}

\subsection*{c)}\textbf{Represente graficamente $\it{f}$.}

	\begin{tikzpicture}[>=latex]
	\begin{axis}[
		axis x line=center,
		axis y line=center,
		xtick={-5,-4,...,5},
		ytick={-5,-4,...,5},
		xlabel={$x$},
		ylabel={$y$},
		xlabel style={below right},
		ylabel style={above left},
		xmin=-5.5,
		xmax=8,
		ymin=-5.5,
		ymax=5.5]
		\addplot[domain=-2:4] {(\x-1)^2-4};
		\draw[dashed] (axis cs:1,0) -- (axis cs:1,-4) -- (axis cs:0,-4);
		\addplot[mark=*] coordinates {(1,-4)};
	\end{axis}
\end{tikzpicture}

\subsection*{d)}\textbf{Indique os valores de x que têm imagem negativa.}

	\begin{tikzpicture}[>=latex]
	\begin{axis}[
		axis x line=center,
		axis y line=center,
		xtick={-5,-4,...,5},
		ytick={-5,-4,...,5},
		xlabel={$x$},
		ylabel={$y$},
		xlabel style={below right},
		ylabel style={above left},
		xmin=-5.5,
		xmax=8,
		ymin=-5.5,
		ymax=5.5]
		\addplot[domain=-1:3] {(\x-1)^2-4};
		\addplot[name path = parab, domain=-1:3] {(\x-1)^2-4};
        \addplot[name path = floor, draw = none] coordinates {(-2,0) (5,0)};
        \addplot[color=gray,opacity=0.5] fill between[of = parab and floor, soft clip={domain = -1:3}];
		\addplot[mark=*] coordinates {(1,-4)};
	\end{axis}
\end{tikzpicture}

\text{$C.S.=\left]-1,3\right[$}

\section*{Exercício 2}\textbf{Considere a função real de variável real definida por $f(x) = -(x + 2)^2 - 1, x \in \mathbb{R}$.}

\subsection*{a)}\textbf{Determine as coordenadas do vértice da parábola representativa do gráfico da função e escreva uma equação do seu eixo de simetria.}

\text{As coordenadadas do $V=(-2,-1)$ e equação do eixo de simetria é $x=-2$}
\subsection*{b)}\textbf{Indique dois objetos diferentes que tenham a mesma imagem por $\it{f}$.}
\[\{1,-10\}\]
\[\{-5,-10\}\]
\subsection*{c)}\textbf{Qual o sentido da concavidade da parábola? Justifique.}

\text{A concavidade da parábola é voltada para baixo porque o coeficiente x ou seja é menor que zero.}

\subsection*{d)}\textbf{Represente graficamente $\it{f}$.}

\begin{tikzpicture}[>=latex]
	\begin{axis}[
		axis x line=center,
		axis y line=center,
		xtick={-6,-5,-4,...,5},
		ytick={-6,-5,-4,...,5},
		xlabel={$x$},
		ylabel={$y$},
		xlabel style={below left},
		ylabel style={above left},
		xmin=-7,
		xmax=8,
		ymin=-7,
		ymax=8]
		\addplot[domain=-5:4] {-(\x+2)^2-1};
		\draw[dashed] (axis cs:-2,0) -- (axis cs:-2,-1) -- (axis cs:0,-1);
		\addplot[mark=*] coordinates {(-2,-1)};
	\end{axis}
\end{tikzpicture}
\subsection*{e)}\textbf{Indique o seu contradomínio.}
\[D'_{f}=\left]-\infty,-1\right]\]

\section*{Exercício 3}\textbf{Resolva em R cada uma das condições:}
\subsection*{a)}\textbf{$9x^2 + 12x + 4 \leq 0$;}
\text{C.A.}
\[9x^2 + 12x + 4 = 0\]
\[\Leftrightarrow x=\frac{-12\pm\sqrt{0}}{18}\]
\[\Leftrightarrow x=\frac{-12}{18}\]
\[\Leftrightarrow x=-\frac{2}{3}\]
\[S=-\frac{2}{3}\]
\[9x^2 + 12x + 4 \leq 0 = -\frac{2}{3}\]
\begin{tikzpicture}[>=latex]
	\begin{axis}[
		axis x line=center,
		axis y line=center,
		xtick={-6,-5,-4,...,5},
		ytick={-6,-5,-4,...,5},
		xlabel={$x$},
		ylabel={$y$},
		xlabel style={below left},
		ylabel style={above left},
		xmin=-7,
		xmax=8,
		ymin=-7,
		ymax=8]
		\addplot[smooth,thick,domain=-5:1] {9*(\x+(2/3))^2};
		\addplot[mark=*] coordinates {(-(2/3),0)};
	\end{axis}
\end{tikzpicture}

\subsection*{b)}\textbf{$-x^2 + 4 < 0$;}

\begin{tikzpicture}[>=latex]
	\begin{axis}[
		axis x line=center,
		axis y line=center,
		xtick={-6,-5,-4,...,5},
		ytick={-6,-5,-4,...,5},
		xlabel={$x$},
		ylabel={$y$},
		xlabel style={below left},
		ylabel style={above left},
		xmin=-7,
		xmax=8,
		ymin=-7,
		ymax=8]
		\addplot[smooth,thick,domain=-6:6] {-(\x)^2+4};
		\addplot[mark=*] coordinates {(-2,0)};
		\addplot[mark=*] coordinates {(2,0)};
	\end{axis}
\end{tikzpicture}
\[C.S.=\left]-\infty,-2\right[\cup\left]2,+\infty\right[\]

\subsection*{c)}\textbf{$-x^2 - 5x \geq 6$;}

\text{C.A.}

\[-x^2 - 5x - 6 = 0\]
\[\Leftrightarrow x=\frac{5\pm 1}{-2}\]
\[\Leftrightarrow x=\frac{5+1}{-2} \lor x=\frac{5-1}{-2}\]
\[\Leftrightarrow x=-3 \lor x=-2\]
\[S=\{-3,-2\}\]

\begin{tikzpicture}[>=latex]
	\begin{axis}[
		axis x line=center,
		axis y line=center,
		xtick={-6,-5,-4,...,5},
		ytick={-6,-5,-4,...,5},
		xlabel={$x$},
		ylabel={$y$},
		xlabel style={below left},
		ylabel style={above left},
		xmin=-7,
		xmax=8,
		ymin=-7,
		ymax=8]
		\addplot[smooth,thick,domain=-6:6] {-(\x)^2-5*x-6};
		\addplot[mark=*] coordinates {(-3,0)};
		\addplot[mark=*] coordinates {(-2,0)};
	\end{axis}
\end{tikzpicture}

\[-x^2 - 5x \geq 6 =  \left[-3,-2\right]\]
~
\subsection*{d)}\textbf{$x^2 + x - 2 > 0$;}

\text{C.A.}

\[x^2 + x - 2 = 0\]
\[\Leftrightarrow x=\frac{-1\pm 3}{2}\]
\[\Leftrightarrow x=\frac{-1+3}{2} \lor x=\frac{-1-3}{2}\]
\[\Leftrightarrow x=1 \lor x=-2\]
\[S=\{-2,1\}\]

\begin{tikzpicture}[>=latex]
	\begin{axis}[
		axis x line=center,
		axis y line=center,
		xtick={-6,-5,-4,...,5},
		ytick={-6,-5,-4,...,5},
		xlabel={$x$},
		ylabel={$y$},
		xlabel style={below left},
		ylabel style={above left},
		xmin=-7,
		xmax=8,
		ymin=-7,
		ymax=8]
		\addplot[smooth,thick,domain=-6:6] {(\x)^2+x-2};
		\addplot[mark=*] coordinates {(1,0)};
		\addplot[mark=*] coordinates {(-2,0)};
	\end{axis}
\end{tikzpicture}

\[x^2 + x - 2 > 0 = \left]-\infty,-2\right[\cup\left]1,+\infty\right[\]

\subsection*{e)}\textbf{$4x^2 + x + 1 < 0$;}

\begin{tikzpicture}[>=latex]
	\begin{axis}[
		axis x line=center,
		axis y line=center,
		xtick={-6,-5,-4,...,5},
		ytick={-6,-5,-4,...,5},
		xlabel={$x$},
		ylabel={$y$},
		xlabel style={below left},
		ylabel style={above left},
		xmin=-7,
		xmax=8,
		ymin=-7,
		ymax=8]
		\addplot[smooth,thick,domain=-6:6] {4*(\x)^2+x+1};
	\end{axis}
\end{tikzpicture}

\text{O $\Delta<0$, portanto não tem zeros e como a concavidade é virada para cima não toca o eixo do x.}


\section*{Exercício 4}\textbf{Determine o domínio das funções definidas por :}
\subsection*{a)}\textbf{$f(x)=\sqrt{2x^2 - 4x}$}

\text{C.A.}
\[2x^2 - 4x=0\]
\[\Leftrightarrow x\left(2x-4\right)=0\]
\[\Leftrightarrow x=0 \lor x=2\]
\[D_{f}=\left]-\infty,0\right]\cup\left[2,+\infty\right[\]

\begin{tikzpicture}
	%\matrix[matrix of math nodes,
	%nodes in empty cells,
	%nodes={text width=2cm,minimum height=8mm,anchor=north east, text centered}, 
	%row 1/.style={nodes={minimum height=5mm}},CE/.list={1,3,5}](S)
	\matrix[matrix of math nodes,
	nodes in empty cells,
	nodes={text width=1cm,minimum height=8mm,anchor=north east, text centered}, 
	row 1/.style={nodes={minimum height=5mm}},CE/.list={1,3,5}](S)
	{
		& & & & & & &  \\
		& & & & & & &  \\
		& & & & & & &  \\
		& & & & & & &  \\
	};
	\fill[top color=brown!20,bottom color=brown!5,middle color=brown!5](S-1-1.south west) [rounded corners=1pt] |- (S-1-6.north east) |- cycle;
	\draw[rounded corners=1pt] (S-1-1.north west) rectangle (S-4-6.south east);
	\draw[ultra thick] (S-1-1.south west) -- (S-1-6.south east);
	\draw (S-2-1.south west) -- (S-2-6.south east);
	\draw (S-3-1.south west) -- (S-3-6.south east);
	\foreach \i in{1,...,6}{
		\draw (S-1-\i.north east) -- (S-4-\i.south east);
		\node at (S-1-1) {$x$};
		\node[anchor=west] at (S-1-2.west) {\(-\infty\)};
		\node[anchor=east] at (S-1-6.east) {\(+\infty\)};
		\node at (S-1-3) {\(0\)};
		\node at (S-1-5) {\(2\)};
		\node at (S-2-1) {\(x\)};
		\node at (S-3-1) {\(2x-4\)};
		\node at (S-2-2) {\(-\)};
		\node at (S-3-2) {\(-\)};
		\node at (S-2-3) {\(0\)};
		\node at (S-3-3) {\(-\)};
		\node at (S-3-6) {\(+\)};
		\node at (S-2-4) {\(+\)};
		\node at (S-3-4) {\(-\)};
		\node at (S-2-5) {\(+\)};
		\node at (S-2-6) {\(+\)};
		\node at (S-3-5) {\(0\)};
		\node at (S-4-1) {\(\left(x\right)\left(2x-4\right)\)};
		\node at (S-4-2) {\(+\)};
		\node at (S-4-3) {\(0\)};
		\node at (S-4-4) {\(-\)};
		\node at (S-4-5) {\(0\)};
		\node at (S-4-6) {\(+	\)};
	}
	\draw[top color=red, fill opacity=.2, decorate,decoration={brace,mirror,amplitude=1.5mm}](S-4-2.south west) to node[midway,fill opacity=1,below]{Crescente} (S-4-3.south east);
	\draw[top color=red, fill opacity=.2, decorate,decoration={brace,mirror,amplitude=1.5mm}](S-4-5.south west) to node[midway,fill opacity=1,below]{Crescente} (S-4-6.south east);
\end{tikzpicture}

\begin{tikzpicture}[>=latex]
	\begin{axis}[
		axis x line=center,
		axis y line=center,
		xtick={-6,-5,-4,...,5},
		ytick={-6,-5,-4,...,5},
		xlabel={$x$},
		ylabel={$y$},
		xlabel style={below left},
		ylabel style={above left},
		xmin=-7,
		xmax=8,
		ymin=-7,
		ymax=8]
		\addplot[smooth,thick,domain=-6:6] {2*(\x)^2-4*(\x))^(0.5)};
		\addplot[mark=*] coordinates {(2,0)};
		\addplot[mark=*] coordinates {(0,0)};
	\end{axis}
\end{tikzpicture}

\subsection*{b)}\textbf{$f(x)=\frac{x}{\sqrt{\left(x-1\right)\left(x-2\right)}}$}
\[D_{f}=\{x \in \mathbb{R}:\left(x-1\right)\left(x-2\right)>0\}\]

\text{C.A.}
\[\left(x-1\right)\left(x-2\right)=0\]
\[\Leftrightarrow x=1 \lor x=2\]
\[D_{f}=\left]-\infty,1\right[\cup\left]2,+\infty\right[\]

\begin{tikzpicture}
	%\matrix[matrix of math nodes,
	%nodes in empty cells,
	%nodes={text width=2cm,minimum height=8mm,anchor=north east, text centered}, 
	%row 1/.style={nodes={minimum height=5mm}},CE/.list={1,3,5}](S)
	\matrix[matrix of math nodes,
	nodes in empty cells,
	nodes={text width=1cm,minimum height=8mm,anchor=north east, text centered}, 
	row 1/.style={nodes={minimum height=5mm}},CE/.list={1,3,5}](S)
	{
		& & & & & & &  \\
		& & & & & & &  \\
		& & & & & & &  \\
		& & & & & & &  \\
	};
	\fill[top color=brown!20,bottom color=brown!5,middle color=brown!5](S-1-1.south west) [rounded corners=1pt] |- (S-1-6.north east) |- cycle;
	\draw[rounded corners=1pt] (S-1-1.north west) rectangle (S-4-6.south east);
	\draw[ultra thick] (S-1-1.south west) -- (S-1-6.south east);
	\draw (S-2-1.south west) -- (S-2-6.south east);
	\draw (S-3-1.south west) -- (S-3-6.south east);
	\foreach \i in{1,...,6}{
		\draw (S-1-\i.north east) -- (S-4-\i.south east);
		\node at (S-1-1) {$x$};
		\node[anchor=west] at (S-1-2.west) {\(-\infty\)};
		\node[anchor=east] at (S-1-6.east) {\(+\infty\)};
		\node at (S-1-3) {\(1\)};
		\node at (S-1-5) {\(2\)};
		\node at (S-2-1) {\(x-1\)};
		\node at (S-3-1) {\(x-2\)};
		\node at (S-2-2) {\(-\)};
		\node at (S-3-2) {\(-\)};
		\node at (S-2-3) {\(0\)};
		\node at (S-3-3) {\(-\)};
		\node at (S-3-6) {\(+\)};
		\node at (S-2-4) {\(+\)};
		\node at (S-3-4) {\(-\)};
		\node at (S-2-5) {\(+\)};
		\node at (S-2-6) {\(+\)};
		\node at (S-3-5) {\(0\)};
		\node at (S-4-1) {\(\left(x-1\right)\left(x-2\right)\)};
		\node at (S-4-2) {\(+\)};
		\node at (S-4-3) {\(0\)};
		\node at (S-4-4) {\(-\)};
		\node at (S-4-5) {\(0\)};
		\node at (S-4-6) {\(+	\)};
	}
	\draw[top color=red, fill opacity=.2, decorate,decoration={brace,mirror,amplitude=1.5mm}](S-4-2.south west) to node[midway,fill opacity=1,below]{Crescente} (S-4-3.south east);
	\draw[top color=red, fill opacity=.2, decorate,decoration={brace,mirror,amplitude=1.5mm}](S-4-5.south west) to node[midway,fill opacity=1,below]{Crescente} (S-4-6.south east);
\end{tikzpicture}

\begin{tikzpicture}[>=latex]
	\begin{axis}[
		axis x line=center,
		axis y line=center,
		xtick={-6,-5,-4,...,5},
		ytick={-6,-5,-4,...,5},
		xlabel={$x$},
		ylabel={$y$},
		xlabel style={below left},
		ylabel style={above left},
		xmin=-7,
		xmax=8,
		ymin=-7,
		ymax=8]
			\addplot[smooth,thick,domain=-6:6]{x/(((\x)-1)*((\x)-2))^(0.5)};
		\addplot[mark=*] coordinates {(1,0)};
		\addplot[mark=*] coordinates {(2,0)};
	\end{axis}
\end{tikzpicture}
\section*{Exercı́cio 5}
\subsection*{a)}
\[f(x)=\begin{cases}
	\text{$2x-x^2$, se $x \leq 2$}\\ 
	\text{$x$, se  $x > 2$}
\end{cases}\]

\begin{tikzpicture}[>=latex]
	\begin{axis}[
		axis x line=center,
		axis y line=center,
		xtick={-7,-6,-5,-4,...,5,6},
		ytick={-7,-6,-5,-4,...,5,6},
		xlabel={$x$},
		ylabel={$y$},
		xlabel style={below right},
		ylabel style={above left},
		xmin=-6.5,
		xmax=8,
		ymin=-6.5,
		ymax=6.5]
		\addplot[domain=-5:2] {2*(\x)-(\x)^2};
		\addplot[domain=2:5] {(\x)};
		\draw[dashed] (axis cs:2,0) -- (axis cs:2,2) -- (axis cs:0,2);
		\addplot[mark=*] coordinates {(2,0)};
		\addplot[mark=*,fill=white] coordinates {(2,2)};
	\end{axis}
\end{tikzpicture}

\subsection*{b)}
\[g(x)=|x^2-4|\]

\begin{tikzpicture}[>=latex]
	\begin{axis}[
		axis x line=center,
		axis y line=center,
		xtick={-7,-6,-5,-4,...,5,6},
		ytick={-7,-6,-5,-4,...,5,6},
		xlabel={$x$},
		ylabel={$y$},
		xlabel style={below right},
		ylabel style={above left},
		xmin=-6.5,
		xmax=8,
		ymin=-6.5,
		ymax=6.5]
		\addplot[domain=-5:5] {abs((\x)^2-4))};
		\addplot[mark=*] coordinates {(2,0)};
		\addplot[mark=*] coordinates {(-2,0)};
		\addplot[mark=*] coordinates {(0,4)};
	\end{axis}
\end{tikzpicture}

\end{document}