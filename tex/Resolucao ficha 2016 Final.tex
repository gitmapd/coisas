\documentclass[a4paper]{article}
\usepackage{amsmath}
\usepackage[utf8]{inputenc}
\usepackage{amsmath}
\usepackage{cancel}
\usepackage[portuguese]{babel}
\usepackage{fancybox}
\usepackage{amssymb}
\usepackage{capt-of}
\usepackage{pgfplots}
\usepackage{tikz}
\usetikzlibrary{matrix}
\usetikzlibrary{calc}
\usetikzlibrary{patterns}
\usetikzlibrary{decorations.pathreplacing}
%\tikzset{
	%	CE/.style={column #1/.style={nodes={text width=24mm}}}
	%}
%\tikzset{
	%	CA/.style={column #1/.style={nodes={text width=35mm}}}
	%}
\tikzset{
	CE/.style={column #1/.style={nodes={text width=25mm}}}
}
\tikzset{
	CA/.style={column #1/.style={nodes={text width=36mm}}}
}
\begin{document}
	\section*{Exercício 1} \textbf{Seja $\left(u_{n}\right)_{n}$ uma sucessão definida por: $u_{n}=\frac{1-3n}{n+1}$}
	\subsection*{a)}
	\textbf{Verifique se $-\frac{14}{5}$ é um dos termos de $\left(u_{n}\right)_{n}$}
	\[\frac{1-3n}{n+1}=-\frac{14}{5}\]
	\[n=19\]
	\[u_{19}=-\frac{14}{5}\]
	\subsection*{b)}
	\textbf{Estude $\left(u_{n}\right)_{n}$ quanto à monotonia}\\
	\\
	\text{$\left(u_{n+1}\right) - \left(u_{n}\right) < 0$ é monótona decrescente}\\
	\text{$\left(u_{n+1}\right) - \left(u_{n}\right) > 0$ é monótona crescente}
	\[\left[\frac{n+1}{n+1}\right]\left[\frac{-3n-2}{n+2}\right] - \left[\frac{1-3n}{n+1}\right]\left[\frac{n+2}{n+2}\right] \]
	
	\[\frac{-3n^2-3n-2n-2}{\left(n+1\right)\left(n+2\right)} - \frac{n+2-3n^2-6n}{\left(n+1\right)\left(n+2\right)}\]
	
	\[\frac{-3n^2-3n-2n-2-n-2+3n^2+6n}{\left(n+1\right)\left(n+2\right)}\]
	
	\[\frac{-4}{\left(n+1\right)\left(n+2\right)}<0, \forall n \in \mathbb{N}\]
	\text{$u_{n}$ é monótona decrescente}
	
	\subsection*{c)}
	\textbf{$\left(u_{n}\right)_{n}$ é uma sucessão convergente? E limitada? Justifique.}
	\[\lim\limits_{n}\frac{1-3n}{n+1} =
	\lim\limits_{n} \frac{\cancel{n}\left(\frac{1}{n}-3\right)}{\cancel{n}\left(1+ \frac{1}{n}\right)} = \frac{\cancelto{0}{\frac{1}{n}}-3}{1+ \cancelto{0}{\frac{1}{n}}} = -3\]
	
	\text{$\left(u_{n}\right)_{n}$ é convergente pois tende para um número real. Toda a sucessão convergente é limitada.}
	
	\text{Como $\left(u_{n}\right)_{n}$ é decrescente sabemos que:}
	\[\frac{1-3n}{n+1} = -3 + \frac{4}{n+1}\]
	\[\frac{4}{n+1} > 0, \text{então qualquer termo será sempre superior a $-3$} \]
	\[-3 < u_{n} < -1, \forall n \in \mathbb{N} \]
\\
\\

\begin{tikzpicture}[scale=1.5][master]
		\begin{axis}[enlargelimits=0.1,]
			\addplot+ [nodes near coords,only marks,point meta=explicit symbolic]
			table [meta=label] {
				x    y   label
				1   -1  $u_{1}$
				2   -1.666 $u_{2}$
				3   -2 $u_{3}$
				4   -2.2 $u_{4}$
				5   -2.333 $u_{5}$
				8   -2.555 $u_{8}$
				10  -2.6363 $u_{10}$
				15  -2.75 $u_{15}$
				19  -2.8 $u_{19}$
			};
		\end{axis}
	\end{tikzpicture}

\section*{Exercício 2}

\textbf{Dê um exemplo concreto de uma sucessão $\left(a_{n}\right)_{n}$ , que verifique em simultâneo as
seguintes afirmações:}\\
\begin{itemize}
	\item \text{$\left(a_{n}\right)_{n}$ é uma sucessão limitada e não monótona}
	\item \text{$\lim\limits_{n} (3a_{n} ) = 0$}\\
	\text{Justifique a sua resposta.}
\end{itemize}
\[a_{n} = \frac{3\left(-1\right)^n}{n}\]




\begin{tikzpicture}[scale=1.5][master]
	\begin{axis}[enlargelimits=0.1,]
		\addplot+ [nodes near coords,only marks,point meta=explicit symbolic]
		table [meta=label] {
			x    y   label
			1   -3  $u_{1}$
			2    1.5 $u_{2}$
			3   -1 $u_{3}$
			4    0.75 $u_{4}$
			5   -0.6 $u_{5}$
			6    0.5 $u_{8}$
			7   -0.428 $u_{10}$
			8    0.375 $u_{15}$
			9   -0.333 $u_{19}$
		};
	\end{axis}
\end{tikzpicture}

\section*{Exercício 3}
\subsection*{a)}
\[\lim\limits_{n}\left(\frac{-7n^3-5n^2+n}{3\sqrt{n^2+1}}\right)\overset{\mathrm{\frac{\infty}{\infty}}}{=}\]

\[\lim\limits_{n}\ \left(\frac{n^{\cancel{3}\left(-7-\cancelto{0}{\frac{5}{n}}+\cancelto{0}{\frac{1}{n^2}}\right)}}{3\cancel{n}\sqrt{1+\cancelto{0}{\frac{1}{n^2}}}}\right)\]

\[= \frac{+\infty\left(-7\right)}{3}=-\infty\]

\subsection*{b)}
\[\lim\limits_{n}\ \left(\sqrt{2n-3}-\sqrt{2n+5}\right)\overset{\mathrm{\infty-\infty}}{=}\]

\[\lim\limits_{n}\ \left(\frac{\left(\sqrt{2n-3}-\sqrt{2n+5}\right)\left(\sqrt{2n-3}+\sqrt{2n+5}\right)}{\left(\sqrt{2n-3}+\sqrt{2n+5}\right)}\right)\]

\[\lim\limits_{n}\ \frac{-8}{\sqrt{2n-3}+\sqrt{2n+5}} = \frac{-8}{+\infty}=0\]

\subsection*{c)}
\[\lim\limits_{n}\ \left(\frac{n-2}{n+1}\right)^{n+3}\overset{\mathrm{1^{\infty}}}{=}\]

\[\lim\limits_{n}\ \left(\frac{\cancel{n}\left(1-\frac{2}{n}\right)}{\cancel{n}\left(1+\frac{1}{n}\right)}\right)^{n} \cdot \lim\limits_{n}\left(\frac{\cancel{n}\left(1-\frac{2}{n}\right)}{\cancel{n}\left(1+\frac{1}{n}\right)}\right)^{3}\]

\[\frac{\lim\limits_{n}\left(1-\cancelto{0}{\frac{2}{n}}\right)^n}{\lim\limits_{n}\left(1+\cancelto{0}{\frac{1}{n}}\right)^n}\cdot \lim\limits_{n}\left(\frac{1-\cancelto{0}{\frac{2}{n}}}{1+\cancelto{0}{\frac{1}{n}}}\right)^{3}=\frac{e^-2}{e}\cdot 1^3=e^{-3}\]

\subsection*{d)}
\[\lim\limits_{n}\left(n^2-\left(-2\right)^nn\right)=+\infty\]

\[\begin{cases}
	\text{Para n par: $\lim\limits_{n}\ \left(n^2-n\right)\overset{\mathrm{\infty-\infty}}{=} \lim\limits_{n}\ \left(n^2\left(1-\cancelto{0}{\frac{1}{n}}\right)\right)=+\infty$}\\ 
	\text{Para n ímpar: $\lim\limits_{n}\ \left(n^2+n\right) = +\infty$}
\end{cases}\]

\section*{Exercício 4}
\subsection*{a)}
\textbf{Considere a função real de variável real definida por $f(x) = \frac{1}{\sqrt{2x+4}}$}
\[D_{f}=\{x \in \mathbb{R}:2x+4 > 0\}=]-2,+\infty[\]
\subsection*{b)}
\textbf{Averigue se o ponto de coordenadas $\left(16,\frac{1}{6}\right)$ pertence ao gráfico de $\it{f}$} .
\[f\left(16\right)=\frac{1}{\sqrt{2\left(16\right)+4}}=\frac{1}{6}=\frac{1}{6}\]
\text{Logo $\left(16,\frac{1}{6}\right)$ pertence a $\it{f}$}

\section*{Exercício 5}
\textbf{Na figura está representada graficamente a função $\it{g}$.}
	
	\begin{tikzpicture}[>=latex]
	\begin{axis}[
		axis x line=center,
		axis y line=center,
		xtick={-5,-4,...,5},
		ytick={-5,-4,...,5},
		xlabel={$x$},
		ylabel={$y$},
		xlabel style={below right},
		ylabel style={above left},
		xmin=-5.5,
		xmax=5.5,
		ymin=-5.5,
		ymax=5.5]
		\addplot[domain=-2:0] {-1/2*(\x)};
		\addplot[domain=0:1] {-(\x)};
		\addplot[domain=1:2] {-2*(\x)+4};
		\addplot[domain=2:3] {(\x)-2};
		\addplot[mark=*,fill=white] coordinates {(1,2)};
		\addplot[mark=*,fill=white] coordinates {(2,0)};
		\addplot[mark=*] coordinates {(2,2)};
		\addplot[mark=*] coordinates {(-2,1)};
		\addplot[mark=*] coordinates {(1,-1)};
		\addplot[mark=*] coordinates {(3,1)};
	\end{axis}
\end{tikzpicture}
\subsection*{a)}\textbf{Indique:}
\begin{itemize}
	\item[i)]\textbf{o domínio e o contradomínio de g;}
	\[D_{g}=[-2,3]\]
	\[D'_{g}=[-1,2]\]
	\item[ii)]\textbf{os zeros de g, se existirem;}
	
	\text{$\it{g}$ só tem um zero que é $x=0$}
	
	\item[iii)]\textbf{um intervalo em que g seja simultaneamente positiva e decrescente;}
	
	\text{$\left[-2,0\right[$ e $\left]1,2\right[$.}
	
	\item[iv)]\textbf{os extremos (máximo e mínimo) absolutos de g, se existirem.}
	\text{Máximo absoluto $2$ e mínimo absoluto $-1$.}
\end{itemize}

\subsection*{b)}\textbf{Indique o número de soluções da condição $\lvert g(x)\rvert=1$.}

\text{Existem 4 soluções: $-2,3,1$ e entre $1$ e $2$.}
\section*{Exercício 6}
\textbf{Considere a funcão quadrática $\it{f}$, de domínio $\mathbb{R}$, definida por  $f(x) = -x^2 + 4x + 5$.}
\subsection*{a)}
\textbf{Escreva a expressão $-x^2 + 4x + 5$ na forma $a\left(x-h\right)^2 + k$ .}
\[f(x)=-\left(x-2\right)^2+9\]
\subsection*{b)}
\textbf{Escreva uma equação do eixo de simetria da parábola representativa do gráfico da função.}
\[x=2\]
\subsection*{c)}
\textbf{Indique dois objetos diferentes que tenham a mesma imagem por $\it{f}$.}
\[x=-1\]
\[x=5\]
\subsection*{d)}
\textbf{Indique, justificando, o contradomínio de $\it{f}$.}
	
	\begin{tikzpicture}[declare function={
		parabola(\x) = -1*\x^2+ 4*\x + 5;
	}]
	\textbf{$D'_{f}=]-\infty,9]$}
	\begin{axis}[
		y axis line style={opacity=0},
		axis x line=middle,
		domain=-2:6,
		scaled ticks=false,
		ytick={\empty},
		xtick={\empty}, 
		xmin = -2,
		xmax = 6,
		ymin = -2,
		ymax = 10,
		]
		\addplot[no marks] {parabola(x)};
	\end{axis}
	\node at (6.3,0.5){$5$};
	\node at (0.5,0.5){$-1$};
	\node at (3.4,5.5){$9$};
	
\end{tikzpicture}

\section*{Exercício 7}
\textbf{Considere a função $\it{h}$ real de domínio $]-1,3]$ definida por:}
\[h(x)=\begin{cases}
	\text{$-2x^2+2$, se $-1 < x \leq 1$}\\ 
	\text{$2x-3$, se $1 < x \leq 3$}
\end{cases}\]


\textbf{Represente graficamente a função $\it{h}$. (Nota: apresente todos os cálculos que efetuar.)}

\begin{tikzpicture}[>=latex]
	\begin{axis}[
		axis x line=center,
		axis y line=center,
		xtick={-5,-4,...,5},
		ytick={-5,-4,...,5},
		xlabel={$x$},
		ylabel={$y$},
		xlabel style={below right},
		ylabel style={above left},
		xmin=-5.5,
		xmax=5.5,
		ymin=-5.5,
		ymax=5.5]
		\addplot[domain=-1:1] {-2*(\x)^2+2};
		\addplot[domain=1:3] {2*(\x)-3};
		\addplot[mark=*,fill=white] coordinates {(-1,0)};
		\addplot[mark=*] coordinates {(1,0)};;
		\addplot[mark=*,fill=white] coordinates {(1,-1)};
		\addplot[mark=*] coordinates {(3,3)};
	\end{axis}
\end{tikzpicture}
\[D_{f}=]-1,3]\]
\[D'_{f}=]-1,3]\]



\section*{Exercício 8}

\subsection*{a)}
\[\textbf{$2x-x^2 \geq 0$}\]

	\begin{tikzpicture}[declare function={
		parabola(\x) = 2*\x - \x^2;
	}]
	\begin{axis}[
		y axis line style={opacity=0},
		axis x line=middle,
		domain=-2:4,
		scaled ticks=false,
		ytick={\empty},
		xtick={\empty}, 
		xmin = -2,
		xmax = 4,
		ymin = -1,
		ymax = 3,
		]
		\addplot[no marks] {parabola(x)};
	\end{axis}
	\node at (5.2,1.1){$2$};
	\node at (1.7,1.1){$0$};
	\node at (3.4,2.2){$+$};
	\node at (0.5,0.5){$-$};
	\node at (6.3,0.5){$-$};
\end{tikzpicture}
\[\text{C.S.=[0,2]}\]

\subsection*{b)}
\[\textbf{$2x^3+3x^2 \leq 2x$}\]
\[\textbf{$\left(x\right)\left(2x^2+3x-2\right)\leq 0$}\]
\begin{tikzpicture}
	%\matrix[matrix of math nodes,
	%nodes in empty cells,
	%nodes={text width=2cm,minimum height=8mm,anchor=north east, text centered}, 
	%row 1/.style={nodes={minimum height=5mm}},CE/.list={1,3,5}](S)
	\matrix[matrix of math nodes,
	nodes in empty cells,
	nodes={text width=1cm,minimum height=8mm,anchor=north east, text centered}, 
	row 1/.style={nodes={minimum height=5mm}},CE/.list={1,3,5}](S)
	{
		& & & & & & & & \\
		& & & & & & & & \\
		& & & & & & & & \\
		& & & & & & & & \\
	};
	\fill[top color=brown!20,bottom color=brown!5,middle color=brown!5](S-1-1.south west) [rounded corners=1pt] |- (S-1-8.north east) |- cycle;
	\draw[rounded corners=1pt] (S-1-1.north west) rectangle (S-4-8.south east);
	\draw[ultra thick] (S-1-1.south west) -- (S-1-8.south east);
	\draw (S-2-1.south west) -- (S-2-8.south east);
	\draw (S-3-1.south west) -- (S-3-8.south east);
	\foreach \i in{1,...,8}{
		\draw (S-1-\i.north east) -- (S-4-\i.south east);
		\node at (S-1-1) {$x$};
		\node[anchor=west] at (S-1-2.west) {\(-\infty\)};
		\node[anchor=east] at (S-1-8.east) {\(+\infty\)};
		\node at (S-1-3) {\(-2\)};
		\node at (S-1-5) {\(0\)};
		\node at (S-1-7) {\(\frac{1}{2}\)};
		\node at (S-2-1) {\(x\)};
		\node at (S-3-1) {\(2x^2+3x-2\)};
		\node at (S-2-2) {\(-\)};
		\node at (S-3-2) {\(+\)};
		\node at (S-2-3) {\(-\)};
		\node at (S-3-3) {\(0\)};
		\node at (S-2-4) {\(-\)};
		\node at (S-3-4) {\(-\)};
		\node at (S-2-5) {\(0\)};
		\node at (S-3-5) {\(-\)};
		\node at (S-2-6) {\(+\)};
		\node at (S-2-7) {\(+\)};
		\node at (S-2-8) {\(+\)};
		\node at (S-3-6) {\(-\)};
		\node at (S-3-7) {\(0\)};
		\node at (S-3-8) {\(+\)};
		\node at (S-4-1) {\(x\left(2x^2+3x-2\right)\)};
		\node at (S-4-2) {\(-\)};
		\node at (S-4-3) {\(0\)};
		\node at (S-4-4) {\(+\)};
		\node at (S-4-5) {\(0\)};
		\node at (S-4-6) {\(-\)};
		\node at (S-4-7) {\(0\)};
		\node at (S-4-8) {\(+\)};
	}
	\draw[top color=red, fill opacity=.2, decorate,decoration={brace,mirror,amplitude=1.5mm}](S-4-2.south west) to node[midway,fill opacity=1,below]{Decrescente} (S-4-3.south east);
	\draw[top color=red, fill opacity=.2, decorate,decoration={brace,mirror,amplitude=1.5mm}](S-4-5.south west) to node[midway,fill opacity=1,below]{Decrescente} (S-4-7.south east);
\end{tikzpicture}
	\[CS]-\infty,-2]\cup [0,\frac{1}{2}]\]

\subsection*{c)}
\[\textbf{$\left(x-2\right)\left(x^2+3\right)\left(4-x\right)>0$}\]


\begin{tikzpicture}
	%\matrix[matrix of math nodes,
	%nodes in empty cells,
	%nodes={text width=2cm,minimum height=8mm,anchor=north east, text centered}, 
	%row 1/.style={nodes={minimum height=5mm}},CE/.list={3,5},CA/.list={1}](S)
	\matrix[matrix of math nodes,
	nodes in empty cells,
	nodes={text width=1cm,minimum height=8mm,anchor=north east, text centered}, 
	row 1/.style={nodes={minimum height=5mm}},CE/.list={1,3,5},CA/.list={1}](S)
	{
		& & & & & & \\
		& & & & & & \\
		& & & & & & \\
		& & & & & & \\
		& & & & & & \\
	};
	\fill[top color=brown!20,bottom color=brown!5,middle color=brown!5](S-1-1.south west) [rounded corners=1pt] |- (S-1-6.north east) |- cycle;
	\draw[rounded corners=1pt] (S-1-1.north west) rectangle (S-5-6.south east);
	\draw[ultra thick] (S-1-1.south west) -- (S-1-6.south east);
	\draw (S-2-1.south west) -- (S-2-6.south east);
	\draw (S-3-1.south west) -- (S-3-6.south east);
	\draw (S-4-1.south west) -- (S-4-6.south east);
	\foreach \i in{1,...,6}{
		\draw (S-1-\i.north east) -- (S-5-\i.south east);
		\node at (S-1-1) {\(x\)};
		\node at (S-2-1) {\(x-2\)};
		\node at (S-3-1) {\(x^2+3\)};
		\node at (S-4-1) {\(4-x\)};
		\node at (S-5-1) {\(\left(x-2\right)\left(x^2+3\right)\left(4-x\right)\)};
		\node[anchor=west] at (S-1-2.west) {\(-\infty\)};
		\node[anchor=east] at (S-1-6.east) {\(+\infty\)};
		\node at (S-1-3) {\(2\)};
		\node at (S-1-5) {\(4\)};
		\node at (S-2-2) {\(-\)};
		\node at (S-2-3) {\(0\)};
		\node at (S-2-4) {\(+\)};
		\node at (S-2-5) {\(+\)};
		\node at (S-2-6) {\(+\)};
		\node at (S-3-2) {\(+\)};
		\node at (S-3-3) {\(+\)};
		\node at (S-3-4) {\(+\)};
		\node at (S-3-5) {\(+\)};
		\node at (S-3-6) {\(+\)};
		\node at (S-4-2) {\(+\)};
		\node at (S-4-3) {\(+\)};
		\node at (S-4-4) {\(+\)};
		\node at (S-4-5) {\(0\)};
		\node at (S-4-6) {\(-\)};
		\node at (S-5-2) {\(-\)};
		\node at (S-5-3) {\(0\)};
		\node at (S-5-4) {\(+\)};
		\node at (S-5-5) {\(0\)};
		\node at (S-5-6) {\(-\)};
	}
	\draw[top color=red, fill opacity=.2, decorate,decoration={brace,mirror,amplitude=1.5mm}](S-5-3.south west) to node[midway,fill opacity=1,below]{Crescente} (S-5-5.south east);
\end{tikzpicture}
\[\text{C.S = $]2,4[$}\]
\end{document}