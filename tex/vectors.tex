\documentclass{article}
\usepackage{tkz-euclide}
\usepackage{pgfplots}
\usepackage{amsmath}
\usepackage{amssymb}
\usetikzlibrary{arrows.meta}
\usepackage[utf8]{inputenc}
\usepackage{cancel}
\usepackage[portuguese]{babel}
\usepackage{fancybox}
\usepackage{amssymb}
\usepackage{capt-of}
\usepackage{tikz}
\usetikzlibrary{matrix}
\usetikzlibrary{calc}
\usetikzlibrary{patterns}
\usetikzlibrary{decorations.pathreplacing}

%\tikzset{
	%	CE/.style={column #1/.style={nodes={text width=24mm}}}
	%}
%\tikzset{
	%	CA/.style={column #1/.style={nodes={text width=35mm}}}
	%}
\tikzset{
	CE/.style={column #1/.style={nodes={text width=25mm}}}
}
\tikzset{
	CA/.style={column #1/.style={nodes={text width=36mm}}}
}

\begin{document}
\section*{Exercício 1}\textbf{Sejam $\it{a}$ e $\it{b}$ números reais. Simplifique a seguinte expressão:}

\textbf{$(a - b)(a + b) + b(b + 2) - 2b$.}

\[a^2-b^2+b^2+2b-2b = a^2\]

\section*{Exercício 2}
%a)
%Resolva, em R, as seguintes condições:
%1 − 2x
%x−1
%≤x−
%.
%2
%3
%b) 4x4 = x2 .
%c) |3 − x| = 2.
	
	\begin{tikzpicture}[vect/.style={->,>={Straight Barb[angle=60:2pt 3]}}]
		\tkzInit[xmin=-6,xmax=6,ymin=-6,ymax=6]
		\tkzDrawX[noticks,>=latex]
		\tkzDrawY[noticks,>=latex]
		\tkzDefPoint(0,0){O}
		\tkzDefPoint(-2,-3){A}
		\tkzDefPoint(-4,1){B}
		\tkzDefPoint(-2,4){AB}
		\tkzDefPoint(0,-4){R}
		\tkzDefPoint(0,2){R2}
		\tkzPointShowCoord[-,xlabel=$-2$,ylabel=$-3$,thin,gray,xstyle={below=4pt}](A)
		\tkzPointShowCoord[-,xlabel=$-4$,ylabel=$1$,thin,gray,xstyle={below=4pt}](B)
		\tkzDrawSegments[vect,color=blue,line width=1pt](A,B)
		\tkzDrawSegments[vect,color=red,line width=1pt](O,R)
		\tkzDrawSegments[vect,color=green,line width=1pt](O,R2)
	\end{tikzpicture}

\section*{Exercício 5}\textbf{Considere a reta $\it{r}$ definida por $\it{r}: 2x - y + 3 = 0$ e o ponto $\it{P}(1, 1)$.}
\subsection*{a)}\textbf{Escreva a equação reduzida da reta $\it{q}$ que é perpendicular a $\it{r}$ e que passa no ponto $\it{P}$.}

\subsection*{b)}\textbf{Determine a distância do ponto $\it{P}$ à reta $\it{r}$.}


\section*{Exercício 6}\textbf{Determine uma expressão geral das soluções reais da equação $-2\sin{x} -\sqrt{2} = 0$}

\[-2\sin{x} -\sqrt{2} = 0\]

\[\Leftrightarrow \sin{x}=-\frac{\sqrt{2}}{2}\]

\[\Leftrightarrow \sin{x}=\sin(-\frac{\pi}{4})\]
\[\Leftrightarrow x=-\frac{\pi}{4} + 2k\pi, k \in \mathbb{Z} \lor x=\pi-\frac{\pi}{4} + 2k\pi, k \in \mathbb{Z} \]
\[\Leftrightarrow x=-\frac{\pi}{4}+ 2k\pi, k \in \mathbb{Z} \lor x=\frac{5\pi}{4}+ 2k\pi, k \in \mathbb{Z} \]

\section*{Exercício 7}\textbf{Mostre, no domínio em que a expressão é válida, que:}

\textbf{$\frac{\sin{x}\cdot \cos{x}}{\tan{x}} = \cos^2{x}$}
\[\frac{\sin{x}\cdot \cos{x}}{\tan{x}} = \frac{\sin{x}\cdot \cos{x}}{\frac{\sin{x}}{\cos{x}}}=\frac{\cancel{\sin{x}}\cdot \cos^2{x}}{\cancel{\sin{x}}}=\cos^2{x}\]
\text{Como queríamos demonstrar.}

\section*{Exercício 8}\textbf{Resolva, em $\mathbb{R}$, a seguinte inequação fracionária:$\frac{-x + 1}{x^2+1}\geq 0$.}

	\text{C.A.}

	\[-x+1=0 \Leftrightarrow x = 1\]
	\[\underbrace{x^2+1=0}_{Imposs\acute{i}vel}\]

\begin{tikzpicture}
	%\matrix[matrix of math nodes,
	%nodes in empty cells,
	%nodes={text width=2cm,minimum height=8mm,anchor=north east, text centered}, 
	%row 1/.style={nodes={minimum height=5mm}},CE/.list={1,3,5}](S)
	\matrix[matrix of math nodes,
	nodes in empty cells,
	nodes={text width=1cm,minimum height=8mm,anchor=north east, text centered}, 
	row 1/.style={nodes={minimum height=5mm}},CE/.list={1}](S)
	{
		& & & & \\
		& & & & \\
		& & & & \\
		& & & & \\
	};
	\fill[top color=brown!20,bottom color=brown!5,middle color=brown!5](S-1-1.south west) [rounded corners=1pt] |- (S-1-4.north east) |- cycle;
	\draw[rounded corners=1pt] (S-1-1.north west) rectangle (S-4-4.south east);
	\draw[ultra thick] (S-1-1.south west) -- (S-1-4.south east);
	\draw (S-2-1.south west) -- (S-2-4.south east);
	\draw (S-3-1.south west) -- (S-3-4.south east);
	\foreach \i in{1,...,4}{
		\draw (S-1-\i.north east) -- (S-4-\i.south east);
	}
	\node at (S-1-1) {$x$};
	\node[anchor=west] at (S-1-2.west) {\(-\infty\)};
	\node[anchor=east] at (S-1-4.east) {\(+\infty\)};

	\foreach \i/\j in{3/\(1\)}{
	\node[anchor=center] at (S-1-\i.center){\j};	
	
}
\foreach \i/\j in{1/\(-x+1\),2/\(+\),3/\(0\),4/\(-\)}{
	\node[anchor=center] at (S-2-\i.center){\j};	
	
}
\foreach \i/\j in{1/\(x^2+1\),2/\(+\),3/\(+\),4/\(+\)}{
	\node[anchor=center] at (S-3-\i.center){\j};	
	
}
\foreach \i/\j in{1/\(\frac{-x+1}{x^2+1}\),2/\(+\),3/\(0\),4/\(-\)}{
	\node[anchor=center] at (S-4-\i.center){\j};	
	
}
\draw[top color=red, fill opacity=.2, decorate,decoration={brace,mirror,amplitude=1.5mm}](S-4-2.south west) to node[midway,fill opacity=1,below]{Crescente} (S-4-3.south east);
%\draw[top color=red, fill opacity=.2, decorate,decoration={brace,mirror,amplitude=1.5mm}](S-4-4.south west) to node[midway,fill opacity=1,below]{Crescente} (S-4-5.south east);
%\fill[pattern=north west lines] (S-2-2.north west) rectangle (S-4-2.south east);
%\draw[->,>=stealth,shorten <=4mm,shorten >=4mm] (S-3-2.north west)--(S-3-2.south east);
%\draw[->,>=stealth,shorten <=4mm,shorten >=4mm] (S-3-4.south west)--(S-3-4.north east);
\end{tikzpicture}
\[C.S=]-\infty,1]\]

\section*{Exercício 9} \textbf{Seja $\left(u_{n}\right)_{n}$ uma sucessão definida por: $u_{n}=1 + \frac{n+1}{n}$}
\subsection*{a)}
\textbf{Verifique se $\frac{11}{5}$ é um dos termos de $\left(u_{n}\right)_{n}$}
\[1 + \frac{n+1}{n}=\frac{11}{5}\]
\[n=5 \in \mathbb{N}\]

\subsection*{b)}
\textbf{Estude $\left(u_{n}\right)_{n}$ quanto à monotonia}\\
\\
\text{$\left(u_{n+1}\right) - \left(u_{n}\right) < 0$ é monótona decrescente}\\
\text{$\left(u_{n+1}\right) - \left(u_{n}\right) > 0$ é monótona crescente}
\[\left[\frac{n}{n}\right]\left[\frac{2n+3}{n+1}\right] - \left[\frac{2n+1}{n}\right]\left[\frac{n+1}{n+1}\right] \]

\[\frac{2n^2+3n}{\left(n+1\right)\left(n\right)} - \frac{2n^2+2n+n+1}{\left(n+1\right)\left(n\right)}\]

\[\frac{2n^2+3n-2n^2-2n-n-1}{\left(n+1\right)\left(n\right)}\]

\[\frac{-1}{\left(n+1\right)\left(n\right)}<0, \forall n \in \mathbb{N}\]
\text{$u_{n}$ é monótona decrescente}


\subsection*{c)}
\textbf{Diga, justificando, se $\left(u_{n}\right)_{n}$ é uma sucessão convergente e se é uma sucessão limitada.}

\[\lim\limits_{n}\ 1 + \frac{n+1}{n} =
\lim\limits_{n} \ 1 + \lim\limits_{n} \ \frac{\cancel{n}\left(1+\frac{1}{n}\right)}{\cancel{n}\left(1\right)} = 1+\frac{1+ \cancelto{0}{\frac{1}{n}}}{1} = 2\]

\text{$\left(u_{n}\right)_{n}$ é convergente pois tende para um número real. Toda a sucessão convergente é limitada.}

\text{Como $\left(u_{n}\right)_{n}$ é decrescente sabemos que:}
\[1 + \frac{n+1}{n}\]
\[\frac{n+1}{n} > 0\]
\[2 < u_{n} \leq 3, \forall n \in \mathbb{N} \]



\begin{tikzpicture}[scale=1.5][master]
	\begin{axis}[enlargelimits=0.1,]
		\addplot+ [nodes near coords,only marks,point meta=explicit symbolic]
		table [meta=label] {
			x    y   label
			1   3.0  $u_{1}$
			2   2.5 $u_{2}$
			3   2.33 $u_{3}$
			4   2.25 $u_{4}$
			5   2.2 $u_{5}$
			6   2.16 $u_{6}$
			7   2.14 $u_{7}$
			8   2.125 $u_{8}$
			9 2.111111111111111 $u_{9}$
			10 2.1 $u_{10}$
			11 2.090909090909091 $u_{11}$
		};		
		
	\end{axis}
\end{tikzpicture}

\section*{Exercício 10}\textbf{Determine, caso existam, os seguintes limites:}
\subsection*{a)}
\[\lim\limits_{n}\ \frac{2n-5}{\sqrt{4n^2+1}}\overset{\mathrm{\frac{\infty}{\infty}}}{=}\]


\[\lim\limits_{n}\ \frac{\cancel{n}\left(2+\cancelto{0}{\frac{5}{n}}\right)}{\cancel{n}\sqrt{4+\cancelto{0}{\frac{1}{n^2}}}}\]

\[= \frac{2}{\sqrt{4+0}}=1\]

\subsection*{b)}
\[\lim\limits_{n}\ \left(\frac{n+1}{n-2}\right)^{3n}\overset{\mathrm{1^{\infty}}}{=}\]

\[=\lim\limits_{n}\ \left(\frac{1+\frac{1}{n}}{1-\frac{2}{n}}\right)^{3n}\]

\[=\left[\frac{\lim\limits_{n}\ \left(1+\cancelto{0}{\frac{1}{n}}\right)^n}{\lim\limits_{n}\ \left(1-\cancelto{0}{\frac{2}{n}}\right)^n}\right]^{3}\]


\[=\left[\frac{e^{1}}{e^{-2}}\right]^{3}\]

\[=e^{9}\]

\section*{Exercício 11} \textbf{Na figura está representada parte de um gráfico de uma função $\it{f}$ de domínio $\mathbb{R}\setminus\{0\}$.}

\begin{tikzpicture}[>=latex]
	\begin{axis}[
		axis x line=center,
		axis y line=center,
		xtick={-9,-8,-7,-6,-5,-4,...,5},
		ytick={-7,-6,-5,-4,...,5},
		xlabel={$x$},
		ylabel={$y$},
		xlabel style={below right},
		ylabel style={above left},
		xmin=-7,
		xmax=5.5,
		ymin=-7,
		ymax=5.5]
		\addplot[thick, smooth, domain=-6:-2] {-2*(\x + 4)^2+2} node[above left, sloped, pos = .05] {\(\rotatebox{-90}{f}\)};
		\addplot[domain=-2:0]{-2};
		\addplot[domain=0:5]{-2*(\x)+2};
		\addplot[mark=*,fill=white] coordinates {(0,-2)};
		\addplot[mark=*] coordinates {(-2,-2)};
		\addplot[mark=*,fill=white] coordinates {(-2,-6)};;
		\addplot[mark=*,fill=white] coordinates {(1,-1)};
		\addplot[mark=*,fill=white] coordinates {(0,2)};
		\draw[dashed] (axis cs:-4,0) -- (axis cs:-4,2) -- (axis cs:0,2);
		\draw[dashed] (axis cs:-2,0) -- (axis cs:-2,-6);
	\end{axis}
\end{tikzpicture}

\textbf{Indique:}
\subsection*{a)}\textbf{$\lim_{x \to 0^{-}} f(x)$}
\[\lim_{x \to 0^{-}} f(x)=-2\]
\subsection*{b)}\textbf{$\lim_{x \to -2^{-}} f(x)$}
\[\lim_{x \to -2^{-}} f(x)=-6\]

\subsection*{c)}\textbf{$\lim_{x \to +\infty} f(x)$}
\[\lim_{x \to +\infty} f(x)=-\infty\]

\section*{Exercício 12}\textbf{Considere a função real, de variável real, definida por $f (x) = 10 - 2^{x - 1}$.}

\subsection*{a)}\textbf{Determine o domínio e o contradomínio da função $\it{f}$.}
\[D_{f}=\mathbb{R}\]
\[D'_{f}=10-2^{x-1}<10=]-\infty,10[\]

\subsection*{b)} \textbf{Caracterize a função inversa da função $\it{f}$ .}


\[f^{-1}=\log_{2}(10-x)+1\]
\[f^{-1}:]-\infty,10[\rightarrow \mathbb{R}\]
\[x\rightarrowtail \log_{2}(10-x)+1\]
\subsection*{c)} \textbf{Resolva em $\mathbb{R}$ a seguinte equação: $f(x) = -6$.}

\[10-2^{x-1}=-6\]
\[\Leftrightarrow 2^{x-1}=2^4\]
\[\Leftrightarrow x=5\]

\section*{Exercício 13}\textbf{Considere a função $\it{f}$ definida por $f(x) = -\frac{x^4}{4} + 2x^2$. Determine, na forma reduzida,a equação da reta tangente ao gráfico de $\it{f}$ no ponto de abcissa $1$.}

\[f'(1)=4x-x^3=3\]
\[f(1)=-\frac{1^4}{4}+2\cdot1^2=\frac{7}{4}\]
\[y-f(1)=f'(1)\cdot(x-1)\]
\[\Leftrightarrow y-\frac{7}{4}=3x-3\]
\[\Leftrightarrow y=3x-\frac{5}{4}\]




\end{document}
