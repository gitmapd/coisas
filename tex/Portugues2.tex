\documentclass[twocolumn]{article}
\usepackage{tabularray}
\usepackage{lipsum}
\usepackage{stfloats} 
\usepackage{xcolor}
\usepackage{soulutf8}
\usepackage{bbding}
\newcommand\itemcolor[1]{\item[\textcolor{blue}#1]}
\setulcolor{green} 
\begin{document}

\begin{tblr}{|c|c|c|c|c|}
\hline
\SetCell[r=2]{c} Orações subordinadas adverbiais
	& \SetCell[c=4]{c} Classificação
	  \SetCell[r=4]{c} & \\
\hline
	& comparativa & consecutiva & concessiva & causal \\
\hline
	 \parbox{10cm}{\textcolor{blue}{a.}Foste de tal forma simpática \ul{que me emprestaste o teu computador para o trabalho.}}& &  \CheckmarkBold & &   \\
\hline
	\parbox{10cm}{\textcolor{blue}{b.}\ul{Como te dedicaste,} foste capaz de uma grande prestação.}& \CheckmarkBold & &  \\
\hline
	\parbox{10cm}{\textcolor{blue}{c.}\ul{Apesar de ser ainda madrugada,} vou correr.}& & & \CheckmarkBold \\
\hline
	\parbox{10cm}{\textcolor{blue}{d.}A Mariana estuda com tanto empenho \ul{como nada com dedicação.}} & \CheckmarkBold & &  \\
\hline
	\parbox{10cm}{\textcolor{blue}{e.}Os citrinos são importantes para a saúde, \ul{visto que contêm vitamina C.}} & & & & \CheckmarkBold \\
\hline
	\parbox{10cm}{\textcolor{blue}{f.}Acompanho-te à consulta, \ul{embora esteja um pouco atrasado}.} & & &  \CheckmarkBold\\
\hline
\end{tblr}
\end{document}