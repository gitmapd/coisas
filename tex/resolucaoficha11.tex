% !TeX program = lualatex
\documentclass[a4paper]{article}
\usepackage{amsmath}
\usepackage[utf8]{inputenc}
\usepackage{amsmath}
\usepackage{cancel}
\usepackage[portuguese]{babel}
\usepackage{fancybox}
\usepackage{amssymb}
\usepackage{capt-of}
\usepackage{pgfplots}
\usepackage{tikz}
\usepackage{polynom}
\usepackage{adjustbox}
\usepackage{pgfplots}
\usepgfplotslibrary{fillbetween}
\usetikzlibrary{matrix}
\usetikzlibrary{calc}
\usetikzlibrary{patterns}
\usetikzlibrary{decorations.pathreplacing}

\tikzset{
	CE/.style={column #1/.style={nodes={text width=43mm}}}
}
%\tikzset{
	%	CA/.style={column #1/.style={nodes={text width=15mm}}}
	%}

\usepackage{luacode}
\begin{luacode}
	-- Computes the roots of x^2+bx+c=0
	-- Returns nothing if they aren't real
	function roots(b, c)
	local delta = b*b-4*c
	if delta > 0 then
	deltasq = math.sqrt(delta)
	local r1 = math.round((-b-deltasq)/2)
	local r2 = math.round((-b+deltasq)/2)
	return r1, r2
	end
	end
	
	-- Outputs x^2+bx+c in developed form to LaTeX
	function display_polynome(b, c)
	p = "x^2"
	if b > 0 then
	p = p .. "+" .. tostring(b) .. "x"
	elseif b < 0 then
	p = p .. tostring(b) .. "x"
	end
	if c > 0 then
	p = p .. "+" .. tostring(c)
	elseif c < 0 then
	p = p .. tostring(c)
	end
	tex.print(p)
	end
	
	-- Outputs x^2+bx+c in factorized form to LaTeX
	function display_factorized_polynom(b, c) 
	r1, r2 = roots(b, c)
	p = "(x"
	if r1 > 0 then
	p = p .. "-" .. tostring(r1) .. ")(x"
	elseif r1 < 0 then
	p = p .. "+" .. tostring(-r1) .. ")(x"
	end
	if r2 > 0 then
	p = p .. "-" .. tostring(r2) .. ")"
	elseif r2 < 0 then
	p = p .. "+" .. tostring(-r2) .. ")"
	end
	tex.print(p)
	end
\end{luacode}

\begin{document}
	\section*{Exercício 1}\textbf{Considere a função real, de variável real, definida por}
	
	\begin{center}
		\textbf{$f(x) = 1 - 3^x$.}
	\end{center}
	\subsection*{a)}\textbf{Determine o domínio e o contradomínio de $\it{f}$.}
	\[D_{f}=\mathbb{R}\]
	\[D'_{f}= \left]-\infty,1\right[\]
	\subsection*{b)}\textbf{Resolva, em $\mathbb{R}$, cada uma das seguintes condições:}
	\begin{itemize}
		\item[i)] \textbf{$f(x)=0$;}
		\[1-3^x=0\]
		\[\Leftrightarrow x=\log_{3}{1}\]
		\[\Leftrightarrow x=0\]
		\item[ii)] \textbf{$f(x)=-26$;}
		\[1-3^x=-26\]
		\[\Leftrightarrow x=\log_{3}{27}\]
		\[\Leftrightarrow x=3\]
		\item[iii)] \textbf{$f(x)<-8$.}
		\[1-3^x<-8\]
		\[\Leftrightarrow x>\log_{3}{9}\]
		\[\Leftrightarrow x>2\]
	\end{itemize}
	\section*{Exercício 2}\textbf{Resolva, em $\mathbb{R}$, cada uma das seguintes condições:}
	
	\subsection*{a)} \textbf{$x^2 \cdot 2^x - \frac{1}{4} \cdot 2^x =0$;}
	\[2^x\left(x^2-\frac{1}{4}\right)=0\]
	\[\underbrace{2^x=0}_{Impossível} \lor x=-\frac{1}{2} \lor x=\frac{1}{2}\]
	\[C.S.=\{-\frac{1}{2},\frac{1}{2}\}\]
	\subsection*{b)} \textbf{$3^{2-x} \leq 27^{-x}$;}
	\[3^{2-x} \leq 3^{-3x}\]
	\[x \leq -1\]
	\[C.S.=\left]-\infty,-1\right[\]
	
\subsection*{c)} \textbf{$10^{x^2-3x} > 0.01$;}
\[10^{x^2-3x} > 10^{-2}\]
\[x^2-3x + 2 > 0\]
\text{C.A.}
\[x^2-3x + 2=0\]
\[x =1 \lor x=2\]
\[C.S.=\left]-\infty,1\right[\cup\left]2,+\infty\right[\]

\end{document}
