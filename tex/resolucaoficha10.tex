% !TeX program = lualatex
\documentclass[a4paper]{article}
\usepackage{amsmath}
\usepackage[utf8]{inputenc}
\usepackage{amsmath}
\usepackage{cancel}
\usepackage[portuguese]{babel}
\usepackage{fancybox}
\usepackage{amssymb}
\usepackage{capt-of}
\usepackage{pgfplots}
\usepackage{tikz}
\usepackage{polynom}
\usepackage{adjustbox}
\usepackage{pgfplots}
\usepgfplotslibrary{fillbetween}
\usetikzlibrary{matrix}
\usetikzlibrary{calc}
\usetikzlibrary{patterns}
\usetikzlibrary{decorations.pathreplacing}

\tikzset{
	CE/.style={column #1/.style={nodes={text width=43mm}}}
}
%\tikzset{
	%	CA/.style={column #1/.style={nodes={text width=15mm}}}
	%}

\usepackage{luacode}
\begin{luacode}
	-- Computes the roots of x^2+bx+c=0
	-- Returns nothing if they aren't real
	function roots(b, c)
	local delta = b*b-4*c
	if delta > 0 then
	deltasq = math.sqrt(delta)
	local r1 = math.round((-b-deltasq)/2)
	local r2 = math.round((-b+deltasq)/2)
	return r1, r2
	end
	end
	
	-- Outputs x^2+bx+c in developed form to LaTeX
	function display_polynome(b, c)
	p = "x^2"
	if b > 0 then
	p = p .. "+" .. tostring(b) .. "x"
	elseif b < 0 then
	p = p .. tostring(b) .. "x"
	end
	if c > 0 then
	p = p .. "+" .. tostring(c)
	elseif c < 0 then
	p = p .. tostring(c)
	end
	tex.print(p)
	end
	
	-- Outputs x^2+bx+c in factorized form to LaTeX
	function display_factorized_polynom(b, c) 
	r1, r2 = roots(b, c)
	p = "(x"
	if r1 > 0 then
	p = p .. "-" .. tostring(r1) .. ")(x"
	elseif r1 < 0 then
	p = p .. "+" .. tostring(-r1) .. ")(x"
	end
	if r2 > 0 then
	p = p .. "-" .. tostring(r2) .. ")"
	elseif r2 < 0 then
	p = p .. "+" .. tostring(-r2) .. ")"
	end
	tex.print(p)
	end
\end{luacode}

\begin{document}
	
	\begin{tikzpicture}
	%\matrix[matrix of math nodes,
	%nodes in empty cells,
	%nodes={text width=2cm,minimum height=8mm,anchor=north east, text centered}, 
	%row 1/.style={nodes={minimum height=5mm}},CE/.list={1,3,5}](S)
	\matrix[matrix of math nodes,
	nodes in empty cells,
	nodes={text width=1cm,minimum height=8mm,anchor=north east, text centered}, 
	row 1/.style={nodes={minimum height=5mm}},CE/.list={1}](S)
	{
		& & & & & & & &\\
		& & & & & & & &\\
		& & & & & & & &\\
		& & & & & & & &\\
	};
	\fill[top color=brown!20,bottom color=brown!5,middle color=brown!5](S-1-1.south west) [rounded corners=1pt] |- (S-1-6.north east) |- cycle;
	\draw[rounded corners=1pt] (S-1-1.north west) rectangle (S-4-6.south east);
	\draw[ultra thick] (S-1-1.south west) -- (S-1-6.south east);
	\draw (S-2-1.south west) -- (S-2-6.south east);
	\draw (S-3-1.south west) -- (S-3-6.south east);
	\draw (S-4-1.south west) -- (S-4-6.south east);
	\foreach \i in{1,...,6}{
		\draw (S-1-\i.north east) -- (S-4-\i.south east);
		\node at (S-1-1) {$x$};
		\node[anchor=west] at (S-1-2.west) {\(-\infty\)};
		\node[anchor=east] at (S-1-6.east) {\(+\infty\)};
		\node at (S-1-3) {\(3\)};
		\node at (S-1-5) {\(5\)};
		\node at (S-2-1) {\(3x-9\)};
		\node at (S-3-1) {\(5-x\)};
		\node at (S-4-1) {\(\frac{3x-9}{5-x}\)};
		\node at (S-2-2) {\(-\)};
		\node at (S-2-3) {\(0\)};
		\node at (S-2-4) {\(+\)};
		\node at (S-2-6) {\(+\)};
		\node at (S-3-2) {\(+\)};
		\node at (S-3-3) {\(+\)};
		\node at (S-3-4) {\(+\)};
		\node at (S-3-6) {\(-\)};
		\node at (S-4-2) {\(-\)};
		\node at (S-4-3) {\(0\)};
		\node at (S-4-4) {\(+\)};
		\node at (S-4-6) {\(-\)};
	}
	%\draw[top color=red, fill opacity=.2, decorate,decoration={brace,mirror,amplitude=1.5mm}](S-4-3.south west) to node[midway,fill opacity=1,below]{Decrescente} (S-4-5.south east);
	\draw[top color=red, fill opacity=.2, decorate,decoration={brace,mirror,amplitude=1.5mm}](S-4-3.south west) to node[midway,fill opacity=1,below]{Crescente} (S-4-5.south east);
	\fill[pattern=north west lines] (S-2-5.north west) rectangle (S-4-5.south east);
\end{tikzpicture}
		\section*{Exercício 2}\textbf{Determine, sob a forma de intervalo ou união de intervalos, o conjunto de números reais que verificam a condição:}\textbf{$\frac{x^2-49}{x^2+6x-7}\leq 0$.}

\text{C.A.}
		\[x^2-49=0\]
		\[\Leftrightarrow x=-7 \lor x=7\]
		\[x^2+x-7=0\]
		\[\Leftrightarrow x=-7 \lor x=1\]
		
		\[C.S=\left]1,7\right]\]
	\begin{tikzpicture}
		%\matrix[matrix of math nodes,
		%nodes in empty cells,
		%nodes={text width=2cm,minimum height=8mm,anchor=north east, text centered}, 
		%row 1/.style={nodes={minimum height=5mm}},CE/.list={1,3,5}](S)
		\matrix[matrix of math nodes,
		nodes in empty cells,
		nodes={text width=1cm,minimum height=8mm,anchor=north east, text centered}, 
		row 1/.style={nodes={minimum height=5mm}},CE/.list={1}](S)
		{
			& & & & & & & &\\
			& & & & & & & &\\
			& & & & & & & &\\
			& & & & & & & &\\
		};
		\fill[top color=brown!20,bottom color=brown!5,middle color=brown!5](S-1-1.south west) [rounded corners=1pt] |- (S-1-8.north east) |- cycle;
		\draw[rounded corners=1pt] (S-1-1.north west) rectangle (S-4-8.south east);
		\draw[ultra thick] (S-1-1.south west) -- (S-1-8.south east);
		\draw (S-2-1.south west) -- (S-2-8.south east);
		\draw (S-3-1.south west) -- (S-3-8.south east);
		\draw (S-4-1.south west) -- (S-4-8.south east);
		\foreach \i in{1,...,8}{
			\draw (S-1-\i.north east) -- (S-4-\i.south east);
			\node at (S-1-1) {$x$};
			\node[anchor=west] at (S-1-2.west) {\(-\infty\)};
			\node[anchor=east] at (S-1-8.east) {\(+\infty\)};
			\node at (S-1-3) {\(-7\)};
			\node at (S-1-5) {\(1\)};
			\node at (S-1-7) {\(7\)};
			\node at (S-2-1) {\(x^2-49\)};
			\node at (S-3-1) {\(x^2+6x-7\)};
			\node at (S-4-1) {\(\frac{x^2-49}{x^2+6x-7}\)};
			\node at (S-2-2) {\(+\)};
			\node at (S-2-3) {\(0\)};
			\node at (S-2-4) {\(-\)};
			\node at (S-2-5) {\(-\)};
			\node at (S-2-6) {\(-\)};
			\node at (S-2-7) {\(0\)};
			\node at (S-2-8) {\(+\)};
			\node at (S-3-2) {\(+\)};
			\node at (S-3-3) {\(0\)};
			\node at (S-3-4) {\(-\)};
			\node at (S-3-5) {\(0\)};
			\node at (S-3-6) {\(+\)};
			\node at (S-3-7) {\(+\)};
			\node at (S-3-8) {\(+\)};
			\node at (S-4-2) {\(+\)};
			\node at (S-4-3) {\(0\)};
			\node at (S-4-4) {\(+\)};
			\node at (S-4-5) {\(SS\)};
			\node at (S-4-6) {\(-\)};
			\node at (S-4-7) {\(0\)};
			\node at (S-4-8) {\(+\)};
		}
		%\draw[top color=red, fill opacity=.2, decorate,decoration={brace,mirror,amplitude=1.5mm}](S-4-3.south west) to node[midway,fill opacity=1,below]{Decrescente} (S-4-5.south east);
		\draw[top color=red, fill opacity=.2, decorate,decoration={brace,mirror,amplitude=1.5mm}](S-4-5.south west) to node[midway,fill opacity=1,below]{Decrescente} (S-4-7.south east);
	\end{tikzpicture}
	\section*{Exercício 1}\textbf{Resolva, em $\mathbb{R}$, cada uma das sequintes inequações:}
	\subsection*{a)}\textbf{$x^3>x^2$}
	
	\text{C.A.}
	\[x^3>x^2\]
	\[\Leftrightarrow x^3-x^2>0\]
	\[\Leftrightarrow x^2\left(x-1\right)>0\]
	\[\Leftrightarrow x>0 \lor x>1\]
	\[C.S=\left]1,+\infty\right[\]
	
	\subsection*{b)}\textbf{$x^3+x^2-2x>0$}
	
	\text{C.A.}
	\[x^3+x^2-2x=0\]
	\[\Leftrightarrow x\left(x^2+x-2\right)=0\]
	\[\Leftrightarrow x=0 \lor x=1 \lor x=-2\]
	\[C.S=\left]-2,0\right[\cup \left]1.+\infty\right[\]	
	\begin{tikzpicture}
		%\matrix[matrix of math nodes,
		%nodes in empty cells,
		%nodes={text width=2cm,minimum height=8mm,anchor=north east, text centered}, 
		%row 1/.style={nodes={minimum height=5mm}},CE/.list={1,3,5}](S)
		\matrix[matrix of math nodes,
		nodes in empty cells,
		nodes={text width=0.8cm,minimum height=8mm,anchor=north east, text centered}, 
		row 1/.style={nodes={minimum height=5mm}},CE/.list={1}](S)
		{
			& & & & & & & & &  \\
			& & & & & & & & & \\
			& & & & & & & & & \\
			& & & & & & & & & \\
		};
		\fill[top color=brown!20,bottom color=brown!5,middle color=brown!5](S-1-1.south west) [rounded corners=1pt] |- (S-1-8.north east) |- cycle;
		\draw[rounded corners=1pt] (S-1-1.north west) rectangle (S-4-8.south east);
		\draw[ultra thick] (S-1-1.south west) -- (S-1-8.south east);
		\draw (S-2-1.south west) -- (S-2-8.south east);
		\draw (S-3-1.south west) -- (S-3-8.south east);
		\foreach \i in{1,...,8}{
			\draw (S-1-\i.north east) -- (S-4-\i.south east);
			\node at (S-1-1) {$x$};
			\node[anchor=west] at (S-1-2.west) {\(-\infty\)};
			\node[anchor=east] at (S-1-8.east) {\(+\infty\)};
			\node at (S-1-3) {\(-2\)};
			\node at (S-1-5) {\(0\)};
			\node at (S-1-7) {\(1\)};
			\node at (S-2-1) {\(x\)};
			\node at (S-3-1) {\(x^2+x-2\)};
			\node at (S-4-1) {\(\left(x\right)\left(x^2+x-2\right)\)};
			\node at (S-2-2) {\(-\)};
			\node at (S-2-3) {\(-\)};
			\node at (S-2-4) {\(-\)};
			\node at (S-2-5) {\(0\)};
			\node at (S-2-6) {\(+\)};
			\node at (S-2-7) {\(+\)};
			\node at (S-2-8) {\(+\)};
			\node at (S-3-2) {\(+\)};
			\node at (S-3-3) {\(0\)};
			\node at (S-3-4) {\(-\)};
			\node at (S-3-5) {\(-\)};
			\node at (S-3-6) {\(-\)};
			\node at (S-3-7) {\(0\)};
			\node at (S-3-8) {\(+\)};
			\node at (S-4-2) {\(-\)};
			\node at (S-4-3) {\(0\)};
			\node at (S-4-4) {\(+\)};
			\node at (S-4-5) {\(0\)};
			\node at (S-4-6) {\(-\)};
			\node at (S-4-7) {\(0\)};
			\node at (S-4-8) {\(+\)};
		}
		\draw[top color=red, fill opacity=.2, decorate,decoration={brace,mirror,amplitude=1.5mm}](S-4-3.south west) to node[midway,fill opacity=1,below]{Crescente} (S-4-5.south east);
		\draw[top color=red, fill opacity=.2, decorate,decoration={brace,mirror,amplitude=1.5mm}](S-4-7.south west) to node[midway,fill opacity=1,below]{Crescente} (S-4-8.south east);
	\end{tikzpicture}
	\subsection*{c)}\textbf{$\left(x-1\right)\left(4-x^2\right)\left(x^2-4x+6\right)\leq 0$}
	
	\text{C.A.}
	\[\left(x-1\right)\left(4-x^2\right)\left(x^2-4x+6\right)=0\]
	\[\Leftrightarrow x=1 \lor x=2 \lor x=-2 \lor x \in \emptyset \]
	\[C.S=\left[-2,1\right]\cup \left[2,+\infty\right[\]

	\section*{Exercı́cio 2}\textbf{Considere a função polinomial definida em $\mathbb{R}$ por $f(x) = x^3 - x^2 - 4x + 4$.}
	
	\subsection*{a)} \textbf{Usando a regra de Ruffini, mostre que $x^3 - x^2 - 4x + 4 = \left(x - 2\right) \left(x^2 + x - 2\right)$ , para todo $x \in \mathbb{R}$.}
	
	
	\polyhornerscheme[x=2]{x^3-x^2-4x+4}
	
	\subsection*{b)} \textbf{Determine os zeros de $\i{f}$.}
	\[\left(x - 2\right) \left(x^2 + x - 2\right)=0\]
	\[\Leftrightarrow x= 2 \lor x=-2 \lor x=1\]
	
	\subsection*{b)} \textbf{Determine o conjunto de números reais que verificam a condição $f(x)<0$.}
	
	
	\[\left(x - 2\right) \left(x^2 + x - 2\right)<0\]
	\[C.S=\left]-\infty,-2\right[\cup \left]1,2\right[\]
	
	\begin{tikzpicture}
		%\matrix[matrix of math nodes,
		%nodes in empty cells,
		%nodes={text width=2cm,minimum height=8mm,anchor=north east, text centered}, 
		%row 1/.style={nodes={minimum height=5mm}},CE/.list={1,3,5}](S)
		\matrix[matrix of math nodes,
		nodes in empty cells,
		nodes={text width=1cm,minimum height=8mm,anchor=north east, text centered}, 
		row 1/.style={nodes={minimum height=5mm}},CE/.list={1}](S)
		{
			& & & & & & & & &  \\
			& & & & & & & & & \\
			& & & & & & & & & \\
			& & & & & & & & & \\
		};
		\fill[top color=brown!20,bottom color=brown!5,middle color=brown!5](S-1-1.south west) [rounded corners=1pt] |- (S-1-8.north east) |- cycle;
		\draw[rounded corners=1pt] (S-1-1.north west) rectangle (S-4-8.south east);
		\draw[ultra thick] (S-1-1.south west) -- (S-1-8.south east);
		\draw (S-2-1.south west) -- (S-2-8.south east);
		\draw (S-3-1.south west) -- (S-3-8.south east);
		\foreach \i in{1,...,8}{
			\draw (S-1-\i.north east) -- (S-4-\i.south east);
			\node at (S-1-1) {$x$};
			\node[anchor=west] at (S-1-2.west) {\(-\infty\)};
			\node[anchor=east] at (S-1-8.east) {\(+\infty\)};
			\node at (S-1-3) {\(-2\)};
			\node at (S-1-5) {\(1\)};
			\node at (S-1-7) {\(2\)};
			\node at (S-2-1) {\(x-2\)};
			\node at (S-3-1) {\(x^2+x-2\)};
			\node at (S-4-1) {\(\left(x-2\right)\left(x^2+x-2\right)\)};
			\node at (S-2-2) {\(-\)};
			\node at (S-2-3) {\(-\)};
			\node at (S-2-4) {\(-\)};
			\node at (S-2-5) {\(-\)};
			\node at (S-2-6) {\(-\)};
			\node at (S-2-7) {\(0\)};
			\node at (S-2-8) {\(+\)};
			\node at (S-3-2) {\(+\)};
			\node at (S-3-3) {\(0\)};
			\node at (S-3-4) {\(-\)};
			\node at (S-3-5) {\(0\)};
			\node at (S-3-6) {\(+\)};
			\node at (S-3-7) {\(+\)};
			\node at (S-3-8) {\(+\)};
			\node at (S-4-2) {\(-\)};
			\node at (S-4-3) {\(0\)};
			\node at (S-4-4) {\(+\)};
			\node at (S-4-5) {\(0\)};
			\node at (S-4-6) {\(-\)};
			\node at (S-4-7) {\(0\)};
			\node at (S-4-8) {\(+\)};;
		}
		\draw[top color=red, fill opacity=.2, decorate,decoration={brace,mirror,amplitude=1.5mm}](S-4-2.south west) to node[midway,fill opacity=1,below]{Decrescente} (S-4-3.south east);
		\draw[top color=red, fill opacity=.2, decorate,decoration={brace,mirror,amplitude=1.5mm}](S-4-5.south west) to node[midway,fill opacity=1,below]{Decrescente} (S-4-7.south east);
	\end{tikzpicture}
	\section*{Exercício 3}\textbf{Considere o polinómio $p(x) = x^4 - 2x^3 - 2x^2 - 2x - 3$.}
	\subsection*{a)}\textbf{Mostre que $p(x)$ é divisı́vel por $(x + 1)(x - 3)$.}
	\[\polyhornerscheme[x=-1]{x^4-2x^3-2x^2-2x-3}\]
	
	\text{$\left(x+1\right)\left(x^3-3x^2+x-3\right)$}
	\[\polyhornerscheme[x=3]{x^3-3x^2+x-3}\]
	\text{$\left(x+1\right)\left(x-3\right)\left(x^2+1\right)$}
	\subsection*{b)}\textbf{Resolva, em $\mathbb{R}$, a inequação $p(x) > 0$.}
	\[\text{$\left(x+1\right)\left(x-3\right)\left(x^2+1\right)>0$}\]
	
	\text{C.A.}
	\[\text{$\left(x+1\right)\left(x-3\right)\left(x^2+1\right)=0$}\]
	\[\Leftrightarrow x=-1 \lor x=3 \lor x \in \emptyset \]
	\[C.S=\left]-\infty,-1\right[\cup \left]3,+\infty\right[\]
	\begin{tikzpicture}
		%\matrix[matrix of math nodes,
		%nodes in empty cells,
		%nodes={text width=2cm,minimum height=8mm,anchor=north east, text centered}, 
		%row 1/.style={nodes={minimum height=5mm}},CE/.list={1,3,5}](S)
		\matrix[matrix of math nodes,
		nodes in empty cells,
		nodes={text width=1cm,minimum height=8mm,anchor=north east, text centered}, 
		row 1/.style={nodes={minimum height=5mm}},CE/.list={1}](S)
		{
			& & & & & & & \\
			& & & & & & & \\
			& & & & & & & \\
			& & & & & & & \\
			& & & & & & & \\
		};
		\fill[top color=brown!20,bottom color=brown!5,middle color=brown!5](S-1-1.south west) [rounded corners=1pt] |- (S-1-6.north east) |- cycle;
		\draw[rounded corners=1pt] (S-1-1.north west) rectangle (S-5-6.south east);
		\draw[ultra thick] (S-1-1.south west) -- (S-1-6.south east);
		\draw (S-2-1.south west) -- (S-2-6.south east);
		\draw (S-3-1.south west) -- (S-3-6.south east);
		\draw (S-4-1.south west) -- (S-4-6.south east);
		\foreach \i in{1,...,6}{
			\draw (S-1-\i.north east) -- (S-5-\i.south east);
			\node at (S-1-1) {$x$};
			\node[anchor=west] at (S-1-2.west) {\(-\infty\)};
			\node[anchor=east] at (S-1-6.east) {\(+\infty\)};
			\node at (S-1-3) {\(-1\)};
			\node at (S-1-5) {\(3\)};
			\node at (S-2-1) {\(x+1\)};
			\node at (S-3-1) {\(x-3\)};
			\node at (S-4-1) {\(x^2+1\)};
			\node at (S-5-1) {\(\left(x+1\right)\left(x-3\right)\left(x^2+1\right)\)};
			\node at (S-2-2) {\(-\)};
			\node at (S-2-3) {\(0\)};
			\node at (S-2-4) {\(+\)};
			\node at (S-2-5) {\(+\)};
			\node at (S-2-6) {\(+\)};
			\node at (S-3-2) {\(-\)};
			\node at (S-3-3) {\(-\)};
			\node at (S-3-4) {\(-\)};
			\node at (S-3-5) {\(0\)};
			\node at (S-3-6) {\(+\)};
			\node at (S-4-2) {\(+\)};
			\node at (S-4-3) {\(+\)};
			\node at (S-4-4) {\(+\)};
			\node at (S-4-5) {\(+\)};
			\node at (S-4-6) {\(+\)};
			\node at (S-5-2) {\(+\)};
			\node at (S-5-3) {\(0\)};
			\node at (S-5-4) {\(-\)};
			\node at (S-5-5) {\(0\)};
			\node at (S-5-6) {\(+\)};
		}
		\draw[top color=red, fill opacity=.2, decorate,decoration={brace,mirror,amplitude=1.5mm}](S-5-2.south west) to node[midway,fill opacity=1,below]{Decrescente} (S-5-3.south east);
		\draw[top color=red, fill opacity=.2, decorate,decoration={brace,mirror,amplitude=1.5mm}](S-5-5.south west) to node[midway,fill opacity=1,below]{Decrescente} (S-5-6.south east);
	\end{tikzpicture}
\end{document}