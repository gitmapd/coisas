	% !TeX program = lualatex
\documentclass[a4paper]{article}
\usepackage{amsmath}
\usepackage[utf8]{inputenc}
\usepackage{amsmath}
\usepackage{cancel}
\usepackage[portuguese]{babel}
\usepackage{fancybox}
\usepackage{amssymb}
\usepackage{capt-of}
\usepackage{pgfplots}
\usepackage{tikz}
\usepackage{polynom}
\usepackage{adjustbox}
\usepackage{pgfplots}
\usepgfplotslibrary{fillbetween}
\usetikzlibrary{matrix}
\usetikzlibrary{calc}
\usetikzlibrary{patterns}
\usetikzlibrary{decorations.pathreplacing}

\pgfplotsset{every axis/.append style={
		axis x line=middle,    % put the x axis in the middle
		axis y line=middle,    % put the y axis in the middle
		axis line style={<->}, % arrows on the axis
		xlabel={$x$},          % default put x on x-axis
		ylabel={$y$},          % default put y on y-axis
}}

% arrows as stealth fighters
\tikzset{>=stealth}
\tikzset{
	CE/.style={column #1/.style={nodes={text width=43mm}}}
}
%\tikzset{
	%	CA/.style={column #1/.style={nodes={text width=15mm}}}
	%}

\begin{document}
	
	\section*{Exercı́cio 5}\textbf{Determine, para cada uma das funções que se seguem, os intervalos de monotonia e os extremos relativos:}
	
	\subsection*{a)}\textbf{$f(x)=x^2-6x+9$}

	\text{C.A.}
	
	\begin{tikzpicture}
		%\matrix[matrix of math nodes,
		%nodes in empty cells,
		%nodes={text width=2cm,minimum height=8mm,anchor=north east, text centered}, 
		%row 1/.style={nodes={minimum height=5mm}},CE/.list={1,3,5}](S)
		\matrix[matrix of math nodes,
		nodes in empty cells,
		nodes={text width=1cm,minimum height=8mm,anchor=north east, text centered}, 
		row 1/.style={nodes={minimum height=5mm}},CE/.list={1}](S)
		{
			& & & & \\
			& & & & \\
			& & & & \\
		};
		\fill[top color=brown!20,bottom color=brown!5,middle color=brown!5](S-1-1.south west) [rounded corners=1pt] |- (S-1-4.north east) |- cycle;
		\draw[rounded corners=1pt] (S-1-1.north west) rectangle (S-3-4.south east);
		\draw[ultra thick] (S-1-1.south west) -- (S-1-4.south east);
		\draw (S-2-1.south west) -- (S-2-4.south east);
		\draw (S-3-1.south west) -- (S-3-4.south east);
		\foreach \i in{1,...,4}{
			\draw (S-1-\i.north east) -- (S-3-\i.south east);
		}
			\node at (S-1-1) {$x$};
			\node[anchor=west] at (S-1-2.west) {\(-\infty\)};
			\node[anchor=east] at (S-1-4.east) {\(+\infty\)};
%			\node at (S-2-1) {\(1+2\ln{\left(x\right)}\)};
%			\node at (S-3-1) {\(x\)};
%			\node at (S-4-1) {\(\frac{1+2\ln{2}}{x}\)};
%			\node at (S-1-2) {\(0\)};
%			\node at (S-1-4) {\(\frac{1}{\sqrt{e}}\)};
%			\node at (S-2-3) {\(+\)};
%			\node at (S-2-4) {\(0\)};
%			\node at (S-2-5) {\(+\)};
%			\node at (S-3-3) {\(-\)};
%			\node at (S-3-4) {\(+\)};
%			\node at (S-3-5) {\(+\)};
%			\node at (S-4-3) {\(-\)};
%			\node at (S-4-4) {\(0\)};
%			\node at (S-4-5) {\(+\)};
		%}
		\foreach \i/\j in{3/\(3\)}{
			\node[anchor=center] at (S-1-\i.center){\j};	
		
	}
		   \foreach \i/\j in{1/\(f'\),2/\(-\),3/\(0\),4/\(+\)}{
			\node[anchor=center] at (S-2-\i.center){\j};	
			
		}
			   \foreach \i/\j in{1/\(f\),3/\(0\)}{
		\node[anchor=center] at (S-3-\i.center){\j};	
		
	}
		%\draw[top color=red, fill opacity=.2, decorate,decoration={brace,mirror,amplitude=1.5mm}](S-4-2.south west) to node[midway,fill opacity=1,below]{Decrescente} (S-4-3.south east);
		%\draw[top color=red, fill opacity=.2, decorate,decoration={brace,mirror,amplitude=1.5mm}](S-4-4.south west) to node[midway,fill opacity=1,below]{Crescente} (S-4-5.south east);
		%\fill[pattern=north west lines] (S-2-2.north west) rectangle (S-4-2.south east);
		\draw[->,>=stealth,shorten <=4mm,shorten >=4mm] (S-3-2.north west)--(S-3-2.south east);
		\draw[->,>=stealth,shorten <=4mm,shorten >=4mm] (S-3-4.south west)--(S-3-4.north east);
	\end{tikzpicture}

	\text{O mínimo é 3.}

	
\begin{tikzpicture}
		%\matrix[matrix of math nodes,
		%nodes in empty cells,
		%nodes={text width=2cm,minimum height=8mm,anchor=north east, text centered}, 
		%row 1/.style={nodes={minimum height=5mm}},CE/.list={1,3,5}](S)
		\matrix[matrix of math nodes,
		nodes in empty cells,
		nodes={text width=1cm,minimum height=8mm,anchor=north east, text centered}, 
		row 1/.style={nodes={minimum height=5mm}},CE/.list={1}](S)
		{
			& & & & \\
			& & & & \\
			& & & & \\
		};
		\fill[top color=brown!20,bottom color=brown!5,middle color=brown!5](S-1-1.south west) [rounded corners=1pt] |- (S-1-4.north east) |- cycle;
		\draw[rounded corners=1pt] (S-1-1.north west) rectangle (S-3-4.south east);
		\draw[ultra thick] (S-1-1.south west) -- (S-1-4.south east);
		\draw (S-2-1.south west) -- (S-2-4.south east);
		\draw (S-3-1.south west) -- (S-3-4.south east);
		\foreach \i in{1,...,4}{
			\draw (S-1-\i.north east) -- (S-3-\i.south east);
		}
		\node at (S-1-1) {$x$};
		\node[anchor=west] at (S-1-2.west) {\(-\infty\)};
		\node[anchor=east] at (S-1-4.east) {\(+\infty\)};
		%			\node at (S-2-1) {\(1+2\ln{\left(x\right)}\)};
		%			\node at (S-3-1) {\(x\)};
		%			\node at (S-4-1) {\(\frac{1+2\ln{2}}{x}\)};
		%			\node at (S-1-2) {\(0\)};
		%			\node at (S-1-4) {\(\frac{1}{\sqrt{e}}\)};
		%			\node at (S-2-3) {\(+\)};
		%			\node at (S-2-4) {\(0\)};
		%			\node at (S-2-5) {\(+\)};
		%			\node at (S-3-3) {\(-\)};
		%			\node at (S-3-4) {\(+\)};
		%			\node at (S-3-5) {\(+\)};
		%			\node at (S-4-3) {\(-\)};
		%			\node at (S-4-4) {\(0\)};
		%			\node at (S-4-5) {\(+\)};
		%}
	\foreach \i/\j in{3/\(2\)}{
		\node[anchor=center] at (S-1-\i.center){\j};	
		
	}
	\foreach \i/\j in{1/\(f''\),2/\(+\),3/\(0\),4/\(+\)}{
		\node[anchor=center] at (S-2-\i.center){\j};	
		
	}
	\foreach \i/\j in{1/\(f\),2/\(\bigcup\),3/\(1\),4/\(\bigcup\)}{
		\node[anchor=center] at (S-3-\i.center){\j};	
		
	}
	%\draw[top color=red, fill opacity=.2, decorate,decoration={brace,mirror,amplitude=1.5mm}](S-4-2.south west) to node[midway,fill opacity=1,below]{Decrescente} (S-4-3.south east);
	%\draw[top color=red, fill opacity=.2, decorate,decoration={brace,mirror,amplitude=1.5mm}](S-4-4.south west) to node[midway,fill opacity=1,below]{Crescente} (S-4-5.south east);
	%\fill[pattern=north west lines] (S-2-2.north west) rectangle (S-4-2.south east);
	%\draw[->,>=stealth,shorten <=4mm,shorten >=4mm] (S-3-2.north west)--(S-3-2.south east);
	%\draw[->,>=stealth,shorten <=4mm,shorten >=4mm] (S-3-4.south west)--(S-3-4.north east);
\end{tikzpicture}
\subsection*{b)}\textbf{$f(x)=x^3-9x^2+3$}

	\begin{tikzpicture}
	%\matrix[matrix of math nodes,
	%nodes in empty cells,
	%nodes={text width=2cm,minimum height=8mm,anchor=north east, text centered}, 
	%row 1/.style={nodes={minimum height=5mm}},CE/.list={1,3,5}](S)
	\matrix[matrix of math nodes,
	nodes in empty cells,
	nodes={text width=1cm,minimum height=8mm,anchor=north east, text centered}, 
	row 1/.style={nodes={minimum height=5mm}},CE/.list={1}](S)
	{
		& & & & & &\\
		& & & & & &\\
		& & & & & &\\
		& & & & & &\\
		& & & & & &\\
	};
	\fill[top color=brown!20,bottom color=brown!5,middle color=brown!5](S-1-1.south west) [rounded corners=1pt] |- (S-1-6.north east) |- cycle;
	\draw[rounded corners=1pt] (S-1-1.north west) rectangle (S-5-6.south east);
	\draw[ultra thick] (S-1-1.south west) -- (S-1-6.south east);
	\draw (S-2-1.south west) -- (S-2-6.south east);
	\draw (S-3-1.south west) -- (S-3-6.south east);
	\draw (S-4-1.south west) -- (S-4-6.south east);
	\foreach \i in{1,...,6}{
		\draw (S-1-\i.north east) -- (S-5-\i.south east);
	}
	\node at (S-1-1) {$x$};
	\node[anchor=west] at (S-1-2.west) {\(-\infty\)};
	\node[anchor=east] at (S-1-6.east) {\(+\infty\)};

\foreach \i/\j in{3/\(0\),5/\(6\)}{
	\node[anchor=center] at (S-1-\i.center){\j};	
	
}
\foreach \i/\j in{1/\(3x\),2/\(-\),3/\(0\),4/\(+\),5/\(+\),6/\(+\)}{
	\node[anchor=center] at (S-2-\i.center){\j};	
	
}
\foreach \i/\j in{1/\(x-6\),2/\(-\),3/\(-\),4/\(-\),5/\(0\),6/\(+\)}{
	\node[anchor=center] at (S-3-\i.center){\j};	
	
}
\foreach \i/\j in{1/\(f'\),2/\(+\),3/\(0\),4/\(-\),5/\(0\),6/\(+\)}{
	\node[anchor=center] at (S-4-\i.center){\j};	
	
}
\foreach \i/\j in{1/\(f\),3/\(3\),5/\(-105\)}{
	\node[anchor=center] at (S-5-\i.center){\j};	
	
}
%\draw[top color=red, fill opacity=.2, decorate,decoration={brace,mirror,amplitude=1.5mm}](S-4-2.south west) to node[midway,fill opacity=1,below]{Decrescente} (S-4-3.south east);
%\draw[top color=red, fill opacity=.2, decorate,decoration={brace,mirror,amplitude=1.5mm}](S-4-4.south west) to node[midway,fill opacity=1,below]{Crescente} (S-4-5.south east);
%\fill[pattern=north west lines] (S-2-2.north west) rectangle (S-4-2.south east);
\draw[->,>=stealth,shorten <=4mm,shorten >=4mm] (S-5-2.south west)--(S-5-2.north east);
\draw[->,>=stealth,shorten <=4mm,shorten >=4mm] (S-5-4.north west)--(S-5-4.south east);
\draw[->,>=stealth,shorten <=4mm,shorten >=4mm] (S-5-6.south west)--(S-5-6.north east);
\end{tikzpicture}


\begin{tikzpicture}
	%\matrix[matrix of math nodes,
	%nodes in empty cells,
	%nodes={text width=2cm,minimum height=8mm,anchor=north east, text centered}, 
	%row 1/.style={nodes={minimum height=5mm}},CE/.list={1,3,5}](S)
	\matrix[matrix of math nodes,
	nodes in empty cells,
	nodes={text width=1cm,minimum height=8mm,anchor=north east, text centered}, 
	row 1/.style={nodes={minimum height=5mm}},CE/.list={1}](S)
	{
		& & & & \\
		& & & & \\
		& & & & \\
	};
	\fill[top color=brown!20,bottom color=brown!5,middle color=brown!5](S-1-1.south west) [rounded corners=1pt] |- (S-1-4.north east) |- cycle;
	\draw[rounded corners=1pt] (S-1-1.north west) rectangle (S-3-4.south east);
	\draw[ultra thick] (S-1-1.south west) -- (S-1-4.south east);
	\draw (S-2-1.south west) -- (S-2-4.south east);
	\draw (S-3-1.south west) -- (S-3-4.south east);
	\foreach \i in{1,...,4}{
		\draw (S-1-\i.north east) -- (S-3-\i.south east);
	}
	\node at (S-1-1) {$x$};
	\node[anchor=west] at (S-1-2.west) {\(-\infty\)};
	\node[anchor=east] at (S-1-4.east) {\(+\infty\)};

\foreach \i/\j in{3/\(3\)}{
	\node[anchor=center] at (S-1-\i.center){\j};	
	
}
\foreach \i/\j in{1/\(f''\),2/\(-\),3/\(0\),4/\(+\)}{
	\node[anchor=center] at (S-2-\i.center){\j};	
	
}
\foreach \i/\j in{1/\(f\),2/\(\bigcap\),3/\(-51\),4/\(\bigcup\)}{
	\node[anchor=center] at (S-3-\i.center){\j};	
	
}
%\draw[top color=red, fill opacity=.2, decorate,decoration={brace,mirror,amplitude=1.5mm}](S-4-2.south west) to node[midway,fill opacity=1,below]{Decrescente} (S-4-3.south east);
%\draw[top color=red, fill opacity=.2, decorate,decoration={brace,mirror,amplitude=1.5mm}](S-4-4.south west) to node[midway,fill opacity=1,below]{Crescente} (S-4-5.south east);
%\fill[pattern=north west lines] (S-2-2.north west) rectangle (S-4-2.south east);
%\draw[->,>=stealth,shorten <=4mm,shorten >=4mm] (S-3-2.north west)--(S-3-2.south east);
%\draw[->,>=stealth,shorten <=4mm,shorten >=4mm] (S-3-4.south west)--(S-3-4.north east);
\end{tikzpicture}
\subsection*{c)}\textbf{$f(x)=\frac{2}{x-1}$;}

\subsection*{d)}\textbf{$h(x)=x+\frac{4}{x}$;}

\begin{tikzpicture}
	%\matrix[matrix of math nodes,
	%nodes in empty cells,
	%nodes={text width=2cm,minimum height=8mm,anchor=north east, text centered}, 
	%row 1/.style={nodes={minimum height=5mm}},CE/.list={1,3,5}](S)
	\matrix[matrix of math nodes,
	nodes in empty cells,
	nodes={text width=1cm,minimum height=8mm,anchor=north east, text centered}, 
	row 1/.style={nodes={minimum height=5mm}},CE/.list={1}](S)
	{
		& & & & & & & &\\
		& & & & & & & &\\
		& & & & & & & &\\
		& & & & & & & &\\
		& & & & & & & &\\
	};
	\fill[top color=brown!20,bottom color=brown!5,middle color=brown!5](S-1-1.south west) [rounded corners=1pt] |- (S-1-8.north east) |- cycle;
	\draw[rounded corners=1pt] (S-1-1.north west) rectangle (S-5-8.south east);
	\draw[ultra thick] (S-1-1.south west) -- (S-1-8.south east);
	\draw (S-2-1.south west) -- (S-2-8.south east);
	\draw (S-3-1.south west) -- (S-3-8.south east);
	\draw (S-4-1.south west) -- (S-4-8.south east);
	\foreach \i in{1,...,8}{
		\draw (S-1-\i.north east) -- (S-5-\i.south east);
	}
	\node at (S-1-1) {$x$};
	\node[anchor=west] at (S-1-2.west) {\(-\infty\)};
	\node[anchor=east] at (S-1-8.east) {\(+\infty\)};
	
	\foreach \i/\j in{3/\(-2\),5/\(0\),7/\(2\)}{
		\node[anchor=center] at (S-1-\i.center){\j};	
		
	}
	\foreach \i/\j in{1/\(x^2-4\),2/\(+\),3/\(0\),4/\(-\),6/\(-\),7/\(0\),8/\(+\)}{
		\node[anchor=center] at (S-2-\i.center){\j};	
		
	}
	\foreach \i/\j in{1/\(x^2\),2/\(+\),3/\(+\),4/\(+\),6/\(+\),7/\(+\),8/\(+\)}{
		\node[anchor=center] at (S-3-\i.center){\j};	
		
	}
	\foreach \i/\j in{1/\(f'\),2/\(+\),3/\(0\),4/\(-\),6/\(-\),7/\(0\),8/\(+\)}{
		\node[anchor=center] at (S-4-\i.center){\j};	
		
	}
	\foreach \i/\j in{1/\(f\)}{
		\node[anchor=center] at (S-5-\i.center){\j};	
		
	}
	%\draw[top color=red, fill opacity=.2, decorate,decoration={brace,mirror,amplitude=1.5mm}](S-4-2.south west) to node[midway,fill opacity=1,below]{Decrescente} (S-4-3.south east);
	%\draw[top color=red, fill opacity=.2, decorate,decoration={brace,mirror,amplitude=1.5mm}](S-4-4.south west) to node[midway,fill opacity=1,below]{Crescente} (S-4-5.south east);
	\fill[pattern=north west lines] (S-2-5.north west) rectangle (S-5-5.south east);
	\draw[->,>=stealth,shorten <=4mm,shorten >=4mm] (S-5-2.south west)--(S-5-2.north east);
	\draw[->,>=stealth,shorten <=4mm,shorten >=4mm] (S-5-4.north west)--(S-5-4.south east);
	\draw[->,>=stealth,shorten <=4mm,shorten >=4mm] (S-5-6.north west)--(S-5-6.south east);
	\draw[->,>=stealth,shorten <=4mm,shorten >=4mm] (S-5-8.south west)--(S-5-8.north east);
	\node[anchor=north] at (S-5-3.north){máx.};
	\node[anchor=south] at (S-5-3.south){\(-4\)};
	\node[anchor=north] at (S-5-7.north){min.};
	\node[anchor=south] at (S-5-7.south){\(4\)};
\end{tikzpicture}

\begin{tikzpicture}
	\begin{axis}[
		xmin=-10,xmax=10,
		ymin=-10,
		ymax=10]
		\addplot [red,smooth,thick,domain=0:10] {x+(4/x)};
		\addplot [red,smooth,thick,domain=-15:-0.3] {x+(4/x)};
		\addplot[brown,dashed] expression {x};
		\addplot[green,dashed] expression {0};
	\end{axis}
\end{tikzpicture}

	
\subsection*{e)}\textbf{$i(x)=e^x\left(x-1\right)$;}

	\begin{tikzpicture}
	%\matrix[matrix of math nodes,
	%nodes in empty cells,
	%nodes={text width=2cm,minimum height=8mm,anchor=north east, text centered}, 
	%row 1/.style={nodes={minimum height=5mm}},CE/.list={1,3,5}](S)
	\matrix[matrix of math nodes,
	nodes in empty cells,
	nodes={text width=1cm,minimum height=8mm,anchor=north east, text centered}, 
	row 1/.style={nodes={minimum height=5mm}},CE/.list={1}](S)
	{
		& & & &\\
		& & & &\\
		& & & &\\
	};
	\fill[top color=brown!20,bottom color=brown!5,middle color=brown!5](S-1-1.south west) [rounded corners=1pt] |- (S-1-4.north east) |- cycle;
	\draw[rounded corners=1pt] (S-1-1.north west) rectangle (S-3-4.south east);
	\draw[ultra thick] (S-1-1.south west) -- (S-1-4.south east);
	\draw (S-2-1.south west) -- (S-2-4.south east);
	\draw (S-3-1.south west) -- (S-3-4.south east);
	\foreach \i in{1,...,4}{
		\draw (S-1-\i.north east) -- (S-3-\i.south east);
	}
	\node at (S-1-1) {$x$};
	\node[anchor=west] at (S-1-2.west) {\(-\infty\)};
	\node[anchor=east] at (S-1-4.east) {\(+\infty\)};

\foreach \i/\j in{3/\(0\)}{
	\node[anchor=center] at (S-1-\i.center){\j};	
	
}
\foreach \i/\j in{1/\(f'\),2/\(-\),3/\(0\),4/\(+\)}{
	\node[anchor=center] at (S-2-\i.center){\j};	
	
}
\foreach \i/\j in{1/\(f\),3/\(-1\)}{
	\node[anchor=center] at (S-3-\i.center){\j};	
	
}

%\draw[top color=red, fill opacity=.2, decorate,decoration={brace,mirror,amplitude=1.5mm}](S-4-2.south west) to node[midway,fill opacity=1,below]{Decrescente} (S-4-3.south east);
%\draw[top color=red, fill opacity=.2, decorate,decoration={brace,mirror,amplitude=1.5mm}](S-4-4.south west) to node[midway,fill opacity=1,below]{Crescente} (S-4-5.south east);
%\fill[pattern=north west lines] (S-2-2.north west) rectangle (S-4-2.south east);

\draw[->,>=stealth,shorten <=4mm,shorten >=4mm] (S-3-2.north west)--(S-3-2.south east);
\draw[->,>=stealth,shorten <=4mm,shorten >=4mm] (S-3-4.south west)--(S-3-4.north east);
\end{tikzpicture}


\begin{tikzpicture}
%\matrix[matrix of math nodes,
%nodes in empty cells,
%nodes={text width=2cm,minimum height=8mm,anchor=north east, text centered}, 
%row 1/.style={nodes={minimum height=5mm}},CE/.list={1,3,5}](S)
\matrix[matrix of math nodes,
nodes in empty cells,
nodes={text width=1cm,minimum height=8mm,anchor=north east, text centered}, 
row 1/.style={nodes={minimum height=5mm}},CE/.list={1}](S)
{
	& & & & \\
	& & & & \\
	& & & & \\
};
\fill[top color=brown!20,bottom color=brown!5,middle color=brown!5](S-1-1.south west) [rounded corners=1pt] |- (S-1-4.north east) |- cycle;
\draw[rounded corners=1pt] (S-1-1.north west) rectangle (S-3-4.south east);
\draw[ultra thick] (S-1-1.south west) -- (S-1-4.south east);
\draw (S-2-1.south west) -- (S-2-4.south east);
\draw (S-3-1.south west) -- (S-3-4.south east);
\foreach \i in{1,...,4}{
	\draw (S-1-\i.north east) -- (S-3-\i.south east);
}
\node at (S-1-1) {$x$};
\node[anchor=west] at (S-1-2.west) {\(-\infty\)};
\node[anchor=east] at (S-1-4.east) {\(+\infty\)};

\foreach \i/\j in{3/\(-1\)}{
\node[anchor=center] at (S-1-\i.center){\j};	

}
\foreach \i/\j in{1/\(f''\),2/\(-\),3/\(0\),4/\(+\)}{
\node[anchor=center] at (S-2-\i.center){\j};	

}
\foreach \i/\j in{1/\(f\),2/\(\bigcap\),3/\(-51\),4/\(\bigcup\)}{
\node[anchor=center] at (S-3-\i.center){\j};	

}
%\draw[top color=red, fill opacity=.2, decorate,decoration={brace,mirror,amplitude=1.5mm}](S-4-2.south west) to node[midway,fill opacity=1,below]{Decrescente} (S-4-3.south east);
%\draw[top color=red, fill opacity=.2, decorate,decoration={brace,mirror,amplitude=1.5mm}](S-4-4.south west) to node[midway,fill opacity=1,below]{Crescente} (S-4-5.south east);
%\fill[pattern=north west lines] (S-2-2.north west) rectangle (S-4-2.south east);
%\draw[->,>=stealth,shorten <=4mm,shorten >=4mm] (S-3-2.north west)--(S-3-2.south east);
%\draw[->,>=stealth,shorten <=4mm,shorten >=4mm] (S-3-4.south west)--(S-3-4.north east);
\end{tikzpicture}

\end{document}
