\documentclass{article}
\usepackage{xcolor}
\usepackage{soulutf8}
\usepackage{tikz}
\usepackage{amsmath}
\usepackage{lipsum}
\usepackage{stix}
\usepackage[utf8]{inputenc}
\usepackage[portuguese]{babel}
\newcommand\textbracketred[2]{$\color{red}\underbracket{\text{\color{black}#1}}_{\text{\color{black}#2}}$}
\newcommand\textbracketgreen[2]{$\color{green}\underbracket{\text{\color{black}#1}}_{\text{\color{black}#2}}$}
\newcommand\textbracketbrown[2]{$\color{brown}\underbracket{\text{\color{black}#1}}_{\text{\color{black}#2}}$}

\newcommand\itemcolor[1]{\item[\textcolor{blue}#1]}
\setulcolor{green} 
\begin{document}
	\begin{tikzpicture} 
		\node[rectangle, thick, dotted, draw=blue] (a) at (0,0) {\parbox{50cm}{
				\begin{itemize}
					\itemcolor{a.}Levo o meu cão a passear, \ul{mesmo que} esteja a chover.
					
					Levo o meu cão a passear, \textbracketred{mesmo que}{concessiva} esteja a chover.
					
					\itemcolor{b.}Vou passar o fim de semana na praia, \ul{a não ser que} chova muito.
					
					Vou passar o fim de semana na praia, \textbracketred{a não ser que}{consecutiva} chova muito.
					
					\itemcolor{c.}A mão pediu-me \ul{para} fechar as janelas.
					
					A mão pediu-me \textbracketred{para}{final} fechar as janelas.
					
					\itemcolor{d.}Este tanto calor \ul{que} o alcatrão derreteu.
					
					Este tanto calor \textbracketred{que}{consecutiva} o alcatrão derreteu.
					
					\itemcolor{e.}Tenho de correr, \ul{visto que} já estou atrasado.
					
					Tenho de correr, \textbracketred{visto que}{causal} já estou atrasado.
					
					\itemcolor{f.}Vou limpar o jardim \ul{antes que} anoiteça.
					
					Vou limpar o jardim \textbracketred{antes que}{temporal} anoiteça.					
					
					\itemcolor{g.}Tenho de decorar o meu papel, \ul{de modo a que} a peça corra bem.
					
					Tenho de decorar o meu papel, \textbracketred{de modo a que}{final} a peça corra bem.
					
					\itemcolor{h.}\ul{Assim como} eu adoro o meu cão, \ul{assim também} ele retribui com gestos carinhosos.
					
					\textbracketred{Assim como}{comparativa} eu adoro o meu cão, \textbracketred{assim também}{consecutiva	} ele retribui com gestos carinhosos.
			\end{itemize}}};
		 %\node[circle, fill=blue,text=white, inner sep=1pt, outer sep=0pt] at (a.north) {C};		
		
	\end{tikzpicture}

\begin{tikzpicture} 
	\node[rectangle, thick, dotted, draw=blue] (a) at (0,0) {\parbox{50cm}{
\begin{itemize}
		\itemcolor{a.}A Joana não dormiu, pois está sempre a bocejar.
		
		A Joana não dormiu, \textbracketred{pois}{Coordenada Explicativa} está sempre a bocejar.
		
		\itemcolor{b.}Nas férias, ou vou para a praia ou procuro um lugar calmo no campo.
		
		Nas férias, \textbracketred{ou vou para a praia ou procuro um lugar calmo}{Coordenada Disjuntiva} no campo.
		
		\itemcolor{c.}O David não só arrumou o quarto como também limpou as janelas.
		
		O David não só arrumou o quarto \textbracketred{como também}{Coordenada Copulativa} limpou as janelas.
		
		\itemcolor{d.}Tenho poupado algum dinheiro, logo posso comprar mais uns livros.
		
		Tenho poupado algum dinheiro, \textbracketred{logo}{Coordenada Conclusiva} posso comprar mais uns livros.
		
		\itemcolor{f.}Adoro policiais, mas prefiro romances históricos.

		Adoro policiais, \textbracketred{mas}{Coordenada Adversativa} prefiro romances históricos.
		
	\end{itemize}}}; 
\end{tikzpicture}

\begin{tikzpicture} 
	\node[rectangle, thick, dotted, draw=blue] (a) at (0,0) {\parbox{50cm}{
			\begin{itemize}
				\itemcolor{a.}O Pedro caminhou muito. As pernas doíam-lhe.(\text{\textcolor{blue}{consequência}})
				
				\textbracketgreen{O Pedro caminhou muito}{Subordinante}, \textbracketred{de maneira que as pernas lhe doíam}{Oração Subordinada Adverbial Consecutiva}.
				
				\itemcolor{b.}O Diogo está ciente da resposta certa. O Diogo não corrigiu o amigo. (\text{\textcolor{blue}{contraste}})
				
				\textbracketgreen{O Diogo está ciente da resposta certa}{Subordinante}, \textbracketred{embora não tenha corrigido o amigo.}{Oração Subordinada Adverbial Concessiva}
				
				\itemcolor{c.}Não conto os meus segredos. Não revelo os desabafos dos amigos.(\text{\textcolor{blue}{adição}})
				
				\textbracketgreen{Não conto os meus segredos}{Oração Coordenada}, \textbracketred{mas também não revelo os desabafos dos amigos.}{Oração coordenada copulativa}
				
				\itemcolor{d.}A Sara estava a estudar. O pai estava a cozinhar.(\text{\textcolor{blue}{tempo}})
				
				\textbracketgreen{A Sara estava a estudar}{Subordinate} \textbracketred{enquanto o pai estava a cozinhar.}{Oração Subordinada Adverbial Temporal}
				
				\itemcolor{e.}O professor está com calor. O professor abriu a janela.(\text{\textcolor{blue}{explicativa}})
				
				\textbracketgreen{O professor está com calor}{Subordinante},\textbracketred{pois abriu a janela.}{Oração Coordenada Explicativa}
				
				\itemcolor{f.}O meu cão salta muito alto. Um canguru salta muito alto.(\text{\textcolor{blue}{comparação}})
				
				\textbracketgreen{O meu cão salta muito alto}{Subordinante}, \textbracketred{que nem um canguru.}{Oração Subordinativa Adverbial Comparativa}
				
				\itemcolor{g.}Adorava ver aquela peça de teatro. O espetáculo foi cancelado.(\text{\textcolor{blue}{oposição}})
				
				\textbracketgreen{Adorava ver aquela peça de teatro}{Oração Coordenada},\textbracketred{mas o espetáculo foi cancelado.}{Oração Coordenada Adversativa}
				
				\itemcolor{h.}Tenho de treinar mais. Pretendo ganhar a maratona.(\text{\textcolor{blue}{condição}})
				
				\textbracketgreen{Tenho de treinar}{Subordinante},\textbracketred{se pretendo ganhar a maratona.}{Oração Subordinada Condicional}
								
	\end{itemize}}}; 
\end{tikzpicture}

\begin{tikzpicture} 
	\node[rectangle, thick, dotted, draw=blue] (a) at (0,0) {\parbox{50cm}{
			\begin{itemize}
				\itemcolor{a.}A casota do cão é tão grande que cabem duas pessoas lá dentro.
				
				\textbracketgreen{A casota do cão é}{Subordinante} \textbracketred{\textbracketbrown{tão}{} grande \textbracketbrown{que}{} cabem duas pessoas lá dentro.}{Oração Subordinada Consecutiva}
				
	\end{itemize}}};
	%\node[circle, fill=blue,text=white, inner sep=1pt, outer sep=0pt] at (a.north) {C};		
	
\end{tikzpicture}


\begin{tikzpicture} 
	\node[rectangle, thick, dotted, draw=blue] (a) at (0,0) {\parbox{50cm}{
			\begin{itemize}
				\itemcolor{a.}O Pedro caminhou muito. As pernas doíam-lhe.(\text{\textcolor{blue}{consequência}})
				
				\textbracketgreen{O Pedro caminhou muito}{Subordinante}, \textbracketred{de maneira que as pernas lhe doíam}{Oração Subordinada Adverbial Consecutiva}.
				
				\itemcolor{b.}O Diogo está ciente da resposta certa. O Diogo não corrigiu o amigo. (\text{\textcolor{blue}{contraste}})
				
				\textbracketgreen{O Diogo está ciente da resposta certa}{Subordinante}, \textbracketred{embora não tenha corrigido o amigo.}{Oração Subordinada Adverbial Concessiva}
				
				\itemcolor{c.}Não conto os meus segredos. Não revelo os desabafos dos amigos.(\text{\textcolor{blue}{adição}})
				
				\textbracketgreen{Não conto os meus segredos}{Oração Coordenada}, \textbracketred{mas também não revelo os desabafos dos amigos.}{Oração coordenada copulativa}
				
				\itemcolor{d.}A Sara estava a estudar. O pai estava a cozinhar.(\text{\textcolor{blue}{tempo}})
				
				\textbracketgreen{A Sara estava a estudar}{Subordinate} \textbracketred{enquanto o pai estava a cozinhar.}{Oração Subordinada Adverbial Temporal}
				
				\itemcolor{e.}O professor está com calor. O professor abriu a janela.(\text{\textcolor{blue}{explicativa}})
				
				\textbracketgreen{O professor está com calor}{Subordinante},\textbracketred{pois abriu a janela.}{Oração Coordenada Explicativa}
				
				\itemcolor{f.}O meu cão salta muito alto. Um canguru salta muito alto.(\text{\textcolor{blue}{comparação}})
				
				\textbracketgreen{O meu cão salta muito alto}{Subordinante}, \textbracketred{que nem um canguru.}{Oração Subordinativa Adverbial Comparativa}
				
				\itemcolor{g.}Adorava ver aquela peça de teatro. O espetáculo foi cancelado.(\text{\textcolor{blue}{oposição}})
				
				\textbracketgreen{Adorava ver aquela peça de teatro}{Oração Coordenada},\textbracketred{mas o espetáculo foi cancelado.}{Oração Coordenada Adversativa}
				
				\itemcolor{h.}Tenho de treinar mais. Pretendo ganhar a maratona.(\text{\textcolor{blue}{condição}})
				
				\textbracketgreen{Tenho de treinar}{Subordinante},\textbracketred{se pretendo ganhar a maratona.}{Oração Subordinada Condicional}
				
	\end{itemize}}}; 
\end{tikzpicture}

\end{document}