	% !TeX program = lualatex
\documentclass[a4paper]{article}
\usepackage{amsmath}
\usepackage{amsmath}
\usepackage{cancel}
\usepackage[portuguese]{babel}
\usepackage{fancybox}
\usepackage{amssymb}
\usepackage{capt-of}
\usepackage{pgfplots}
\usepackage{tikz}
\usepackage{polynom}
\usepackage{adjustbox}
\usepackage{pgfplots}
\usepgfplotslibrary{fillbetween}
\usetikzlibrary{matrix}
\usetikzlibrary{calc}
\usetikzlibrary{patterns}
\usetikzlibrary{decorations.pathreplacing}
\pgfplotsset{compat=1.18}
\newcommand*{\qf}{$x=\frac{-b\pm\sqrt{b^2-4ac}}{2a}$}

% arrows as stealth fighters
\tikzset{>=stealth}
\tikzset{
	CE/.style={column #1/.style={nodes={text width=43mm}}}
}
%\tikzset{
	%	CA/.style={column #1/.style={nodes={text width=15mm}}}
	%}

\begin{document}
	\section*{a)}
	\section*{b)}
	\textbf{$\frac{2x^2-5x-7}{x+2}\leq 0$}
	\[D = {x \in \mathbb{R} : x + 2 \neq 0} = \mathbb{R}\setminus\{-2\}\]
	\[ 2x^2-5x-7= (x+1)(x-\frac{7}{2})=0 \Longleftrightarrow x = -1 \lor x = \frac{7}{2} \]

\begin{tikzpicture}
%\matrix[matrix of math nodes,
%nodes in empty cells,
%nodes={text width=2cm,minimum height=8mm,anchor=north east, text centered}, 
%row 1/.style={nodes={minimum height=5mm}},CE/.list={1,3,5}](S)
\matrix[matrix of math nodes,
nodes in empty cells,
nodes={text width=1cm,minimum height=8mm,anchor=north east, text centered}, 
row 1/.style={nodes={minimum height=5mm}},CE/.list={1}](S)
{
	& & & & & & & &\\
	& & & & & & & &\\
	& & & & & & & &\\
	& & & & & & & &\\
};
\fill[top color=brown!20,bottom color=brown!5,middle color=brown!5](S-1-1.south west) [rounded corners=1pt] |- (S-1-8.north east) |- cycle;
\draw[rounded corners=1pt] (S-1-1.north west) rectangle (S-4-8.south east);
\draw[ultra thick] (S-1-1.south west) -- (S-1-8.south east);
\draw (S-2-1.south west) -- (S-2-8.south east);
\draw (S-3-1.south west) -- (S-3-8.south east);
\draw (S-4-1.south west) -- (S-4-8.south east);
\foreach \i in{1,...,8}{
	\draw (S-1-\i.north east) -- (S-4-\i.south east);
}
\node at (S-1-1) {$x$};
\node[anchor=west] at (S-1-2.west) {\(-\infty\)};
\node[anchor=east] at (S-1-8.east) {\(+\infty\)};

\foreach \i/\j in{3/\(-2\),5/\(-1\),7/\(\frac{7}{2}\)}{
	\node[anchor=center] at (S-1-\i.center){\j};	
	
}
\foreach \i/\j in{1/\(2x^2-5x-7\),2/\(+\),4/\(+\),5/\(0\),6/\(-\),7/\(0\),8/\(+\)}{
	\node[anchor=center] at (S-2-\i.center){\j};	
	
}
\foreach \i/\j in{1/\(x+2\),2/\(-\),4/\(+\),5/\(+\),6/\(+\),7/\(+\),8/\(+\)}{
	\node[anchor=center] at (S-3-\i.center){\j};	
	
}
\foreach \i/\j in{1/\(\frac{2x^2-5x-7}{x+2}\),2/\(-\),4/\(+\),5/\(0\),6/\(-\),7/\(0\),8/\(+\)}{
	\node[anchor=center] at (S-4-\i.center){\j};	
	
}
%\foreach \i/\j in{1/\(f\)}{
%	\node[anchor=center] at (S-5-\i.center){\j};	
%	
%}
\draw[top color=red, fill opacity=.2, decorate,decoration={brace,mirror,amplitude=1.5mm}](S-4-2.south west) to node[midway,fill opacity=1,below]{Decrescente} (S-4-3.south east);
\draw[top color=red, fill opacity=.2, decorate,decoration={brace,mirror,amplitude=1.5mm}](S-4-5.south west) to node[midway,fill opacity=1,below]{Decrescente} (S-4-7.south east);
\fill[pattern=north west lines] (S-2-3.north west) rectangle (S-4-3.south east);
%\draw[->,>=stealth,shorten <=4mm,shorten >=4mm] (S-5-2.south west)--(S-5-2.north east);
%\draw[->,>=stealth,shorten <=4mm,shorten >=4mm] (S-5-4.north west)--(S-5-4.south east);
%\draw[->,>=stealth,shorten <=4mm,shorten >=4mm] (S-5-6.north west)--(S-5-6.south east);
%\draw[->,>=stealth,shorten <=4mm,shorten >=4mm] (S-5-8.south west)--(S-5-8.north east);
%\node[anchor=north] at (S-5-3.north){máx.};
%\node[anchor=south] at (S-5-3.south){\(-4\)};
%\node[anchor=north] at (S-5-7.north){min.};
%\node[anchor=south] at (S-5-7.south){\(4\)};
\end{tikzpicture}
\end{document}
