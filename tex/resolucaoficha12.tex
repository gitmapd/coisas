% !TeX program = lualatex
\documentclass[a4paper]{article}
\usepackage{amsmath}
\usepackage[utf8]{inputenc}
\usepackage{amsmath}
\usepackage{cancel}
\usepackage[portuguese]{babel}
\usepackage{fancybox}
\usepackage{amssymb}
\usepackage{capt-of}
\usepackage{pgfplots}
\usepackage{tikz}
\usepackage{polynom}
\usepackage{adjustbox}
\usepackage{pgfplots}
\usepgfplotslibrary{fillbetween}
\usetikzlibrary{matrix}
\usetikzlibrary{calc}
\usetikzlibrary{patterns}
\usetikzlibrary{decorations.pathreplacing}

\tikzset{
	CE/.style={column #1/.style={nodes={text width=43mm}}}
}
%\tikzset{
	%	CA/.style={column #1/.style={nodes={text width=15mm}}}
	%}

\usepackage{luacode}
\begin{luacode}
	-- Computes the roots of x^2+bx+c=0
	-- Returns nothing if they aren't real
	function roots(b, c)
	local delta = b*b-4*c
	if delta > 0 then
	deltasq = math.sqrt(delta)
	local r1 = math.round((-b-deltasq)/2)
	local r2 = math.round((-b+deltasq)/2)
	return r1, r2
	end
	end
	
	-- Outputs x^2+bx+c in developed form to LaTeX
	function display_polynome(b, c)
	p = "x^2"
	if b > 0 then
	p = p .. "+" .. tostring(b) .. "x"
	elseif b < 0 then
	p = p .. tostring(b) .. "x"
	end
	if c > 0 then
	p = p .. "+" .. tostring(c)
	elseif c < 0 then
	p = p .. tostring(c)
	end
	tex.print(p)
	end
	
	-- Outputs x^2+bx+c in factorized form to LaTeX
	function display_factorized_polynom(b, c) 
	r1, r2 = roots(b, c)
	p = "(x"
	if r1 > 0 then
	p = p .. "-" .. tostring(r1) .. ")(x"
	elseif r1 < 0 then
	p = p .. "+" .. tostring(-r1) .. ")(x"
	end
	if r2 > 0 then
	p = p .. "-" .. tostring(r2) .. ")"
	elseif r2 < 0 then
	p = p .. "+" .. tostring(-r2) .. ")"
	end
	tex.print(p)
	end
\end{luacode}

\begin{document}
	\section*{Exercício 1}\textbf{Calcule:}
	\subsection*{a)}\textbf{$\log_{2}\left(\frac{1}{64}\right)$;}
\[\log_{2}\left(\frac{1}{64}\right)=\log_{2}\left(2^6\right)=6\]

\subsection*{b)}\textbf{$\log\left(1000\right)$;}
\[\log\left(1000\right)=\log\left(10^2\right)=2\]

\subsection*{c)}\textbf{$\ln\left(e^3\right)$;}
\[\ln\left(e^3\right)=3\]

\subsection*{d)}\textbf{$\ln\left(\sqrt[5]{e}\right)$;}
\[\ln\left(\sqrt[5]{e} \right)=\ln\left(e^{\frac{1}{5}} \right)=\frac{1}{5}\]

\subsection*{e)}\textbf{$\ln\left(e^2\right)+\ln\left(e^{-10}\right)\ln\left(1\right)$;}
\[\ln\left(e^2\right)+\ln\left(e^{-10}\right)\ln\left(1\right)=-8\]

\subsection*{f)}\textbf{$\log_{3}\left(\frac{\sqrt{27}}{81^8}\right)$;}
\[\log_{3}\left(\frac{\sqrt{27}}{81^8}\right)=\log_{3} 3^{\frac{3}{2}}-\log_{3}3^{32}=\frac{3}{2}-32=-\frac{61}{2}\]

\subsection*{g)}\textbf{$\log_{4}\left(64\right)$;}
\[\log_{4}\left(64\right)=\log_{4}\left(4^3\right)=3\]

\subsection*{h)}\textbf{$\log_{2}\left(\sqrt{32}\right)$;}
\[\log_{2}\left(\sqrt{32}\right)=\log_{2}\left(2^{\frac{5}{2}}\right)=\frac{5}{2}\]

\subsection*{i)}\textbf{$\log_{5}\left(1\right)$;}
\[\log_{5}\left(1\right)=0\]

\section*{Exercı́cio 2}\textbf{Seja $f(x)=\frac{1+2\ln{\left(x\right)}}{x}$.}

\subsection*{a)}\textbf{Determine $D_{f}$.}
\[D_{f}=\{x \in \mathbb{R}: x > 0 \land x \neq 0\}=]0,+\infty[\]

\subsection*{b)}\textbf{Resolva a inequação $f(x) \geq 0$.}

\[\frac{1+2\ln{\left(x\right)}}{x} \geq 0\]

\text{C.A.}
\begin{tikzpicture}
	%\matrix[matrix of math nodes,
	%nodes in empty cells,
	%nodes={text width=2cm,minimum height=8mm,anchor=north east, text centered}, 
	%row 1/.style={nodes={minimum height=5mm}},CE/.list={1,3,5}](S)
	\matrix[matrix of math nodes,
	nodes in empty cells,
	nodes={text width=1cm,minimum height=8mm,anchor=north east, text centered}, 
	row 1/.style={nodes={minimum height=5mm}},CE/.list={1}](S)
	{
		& & & & & & & & \\
		& & & & & & & & \\
		& & & & & & & & \\
		& & & & & & & & \\
	};
	\fill[top color=brown!20,bottom color=brown!5,middle color=brown!5](S-1-1.south west) [rounded corners=1pt] |- (S-1-5.north east) |- cycle;
	\draw[rounded corners=1pt] (S-1-1.north west) rectangle (S-4-5.south east);
	\draw[ultra thick] (S-1-1.south west) -- (S-1-5.south east);
	\draw (S-2-1.south west) -- (S-2-5.south east);
	\draw (S-3-1.south west) -- (S-3-5.south east);
	\foreach \i in{1,...,5}{
		\draw (S-1-\i.north east) -- (S-4-\i.south east);
		\node at (S-1-1) {$x$};
		\node[anchor=east] at (S-1-5.east) {\(+\infty\)};
		\node at (S-2-1) {\(1+2\ln{\left(x\right)}\)};
		\node at (S-3-1) {\(x\)};
		\node at (S-4-1) {\(\frac{1+2\ln{2}}{x}\)};
		\node at (S-1-2) {\(0\)};
		\node at (S-1-4) {\(\frac{1}{\sqrt{e}}\)};
		\node at (S-2-3) {\(-\)};
		\node at (S-2-4) {\(0\)};
		\node at (S-2-5) {\(+\)};
		\node at (S-3-3) {\(+\)};
		\node at (S-3-4) {\(+\)};
		\node at (S-3-5) {\(+\)};
		\node at (S-4-3) {\(-\)};
		\node at (S-4-4) {\(0\)};
		\node at (S-4-5) {\(+\)};
	}
	%\draw[top color=red, fill opacity=.2, decorate,decoration={brace,mirror,amplitude=1.5mm}](S-4-2.south west) to node[midway,fill opacity=1,below]{Decrescente} (S-4-3.south east);
	\draw[top color=red, fill opacity=.2, decorate,decoration={brace,mirror,amplitude=1.5mm}](S-4-4.south west) to node[midway,fill opacity=1,below]{Crescente} (S-4-5.south east);
	\fill[pattern=north west lines] (S-2-2.north west) rectangle (S-4-2.south east);
\end{tikzpicture}
\[C.S.=[\frac{1}{\sqrt{e}},+\infty[\]
\section*{Exercício 3}\textbf{Para cada uma das funções seguintes, determine o domínio, o contradomínio e os zeros.}
\textbf{Caracterize, caso exista, a função inversa.}

\subsection*{a)}\textbf{$m(x)=5 -\log(x+5)$;}
\[D_{m}=\{x \in \mathbb{R}: x > -5\}=]-5,+\infty[\]
\[D'_{m}=\mathbb{R}\]
\[m^{-1}=10^{-x+5}-5\]
\[m^{-1}:\mathbb{R}\rightarrow ]-5,+\infty[\]
\[x\rightarrowtail 10^{-x+5}-5\]

\subsection*{b)}\textbf{$g(x)=3 + \frac{1}{2}\log_{7}(2x-1)$;}
\[D_{g}=\{x \in \mathbb{R}: x > \frac{1}{2}\}=]\frac{1}{2},+\infty[\]
\[D'_{g}=\mathbb{R}\]
\[g^{-1}=\frac{7^{2x-6}+1}{2}\]
\[g^{-1}:\mathbb{R}\rightarrow ]\frac{1}{2},+\infty[\]
\[x\rightarrowtail \frac{7^{2x-6}+1}{2}\]

\subsection*{c)}\textbf{$f(x)=e^{x-3}-2$;}
\[D_{f}=\mathbb{R}\]
\[D'_{f}=]-2,+\infty[\]
\[f^{-1}=\ln{\left(x+2\right)}+3\]
\[f^{-1}:]-2,+\infty[\rightarrow \mathbb{R}\]
\[x\rightarrowtail \ln{\left(x+2\right)}+3\]

\section*{Exercício 4}\textbf{Resolva, em $\mathbb{R}$, cada uma das seguintes condições:}
\subsection*{a)}\textbf{$\ln(x^2-1)=1$;}
\[D=\{x \in \mathbb{R}:x^2 - 1 > 0\}=]-\infty,-1[ \cup ]1.+\infty[\]
\[\ln(x^2-1)=1\Leftrightarrow x^2=1+e \Leftrightarrow x = \pm \sqrt{1+e}\]
\[C.S=\{-\sqrt{1+e},\sqrt{1+e}\}\]
\subsection*{b)}\textbf{$\log_{2}(1-2x) > \log{2}(x)$;}
\[D=\{x \in \mathbb{R}:1-2x > 0 \land x > 0\}=]0,\frac{1}{2}[\]
\[x < \frac{1}{3} \land x \in D\]
\[D \cap ]-\infty,\frac{1}{3}[=]0,\frac{1}{2}[ \cap ]-\infty,\frac{1}{3}[ = ]0,\frac{1}{3}[ \]

\subsection*{c)}\textbf{$\log(1-x^2) < 1$.}
\[D=\{x \in \mathbb{R}:1-x^2 > 0\}=]-1,1[\]
\[\underbrace{x^2 > -9}_{Impossível} \land x \in D\]
\[x \in \mathbb{R} \land x \in D\]
\[D \cap \mathbb{R}=]-1,1[\]

\section*{Exercício 5}\textbf{Considere a função real, de variável real, definida por}
\begin{center}
	\textbf{$f(x) = 1 - 3^x$	}
\end{center}
\subsection*{a)}\textbf{Calcule $f(0) + f(\log_{3}2)$.}
\[f(0) + f(\log_{3}2)=-1\]
\subsection*{b)}\textbf{Caracterize, caso exista, a função inversa $f^{-1}$.}

\[D_{f}=\mathbb{R}\]
\[D'_{f}=]-\infty,1[\]
\[f^{-1}=\log_{3}\left(-x+1\right)\]
\[f^{-1}:]-\infty,1[\rightarrow\mathbb{R}[\]
\[x\rightarrowtail \log_{3}\left(-x+1\right)\]
\end{document}
