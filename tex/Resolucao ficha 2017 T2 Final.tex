% !TeX program = lualatex
\documentclass[a4paper]{article}
\usepackage{amsmath}
\usepackage[utf8]{inputenc}
\usepackage{amsmath}
\usepackage{cancel}
\usepackage[portuguese]{babel}
\usepackage{fancybox}
\usepackage{amssymb}
\usepackage{capt-of}
\usepackage{pgfplots}
\usepackage{tikz}
\usepackage{polynom}
\usepackage{adjustbox}
\usepackage{pgfplots}
\usepgfplotslibrary{fillbetween}
\usetikzlibrary{matrix}
\usetikzlibrary{calc}
\usetikzlibrary{patterns}
\usetikzlibrary{decorations.pathreplacing}

\pgfplotsset{every axis/.append style={
		axis x line=middle,    % put the x axis in the middle
		axis y line=middle,    % put the y axis in the middle
		axis line style={<->}, % arrows on the axis
		xlabel={$x$},          % default put x on x-axis
		ylabel={$y$},          % default put y on y-axis
}}

% arrows as stealth fighters
\tikzset{>=stealth}
\tikzset{
	CE/.style={column #1/.style={nodes={text width=43mm}}}
}
%\tikzset{
	%	CA/.style={column #1/.style={nodes={text width=15mm}}}
	%}

\begin{document}
\section*{Exercício 1} \textbf{Considere a funcão polinomial definida em $\mathbb{R}$ por $p(x)=x^3-3x-2$.}
\subsection*{a)}
\textbf{Mostre, usando a regra de Ruffini, que $p(x)=(x+1)(x^2-x-2)$, para qualquer $ x \in \mathbb{R}$.}


\polyhornerscheme[x=-1]{x^3-3x-2}

\text{Portanto, $p(x)=(x+1)(x^2-x-2)$}

\subsection*{b)}\textbf{Determine, sob a forma de intervalo ou união de intervalos, o conjunto de números reais que verificam a condição $\frac{p(x)}{x} \leq 0$.}\\

\text{$\frac{(x+1)(x^2-x-2)}{x}$}

\begin{tikzpicture}
%\matrix[matrix of math nodes,
%nodes in empty cells,
%nodes={text width=2cm,minimum height=8mm,anchor=north east, text centered}, 
%row 1/.style={nodes={minimum height=5mm}},CE/.list={1,3,5}](S)
\matrix[matrix of math nodes,
nodes in empty cells,
nodes={text width=1cm,minimum height=8mm,anchor=north east, text centered}, 
row 1/.style={nodes={minimum height=5mm}},CE/.list={1}](S)
{
	& & & & & & & &\\
	& & & & & & & &\\
	& & & & & & & &\\
	& & & & & & & &\\
	& & & & & & & &\\
};
\fill[top color=brown!20,bottom color=brown!5,middle color=brown!5](S-1-1.south west) [rounded corners=1pt] |- (S-1-8.north east) |- cycle;
\draw[rounded corners=1pt] (S-1-1.north west) rectangle (S-5-8.south east);
\draw[ultra thick] (S-1-1.south west) -- (S-1-8.south east);
\draw (S-2-1.south west) -- (S-2-8.south east);
\draw (S-3-1.south west) -- (S-3-8.south east);
\draw (S-4-1.south west) -- (S-4-8.south east);
\foreach \i in{1,...,8}{
	\draw (S-1-\i.north east) -- (S-5-\i.south east);
}
\node at (S-1-1) {$x$};
\node[anchor=west] at (S-1-2.west) {\(-\infty\)};
\node[anchor=east] at (S-1-8.east) {\(+\infty\)};

\foreach \i/\j in{3/\(-1\),5/\(0\),7/\(2\)}{
	\node[anchor=center] at (S-1-\i.center){\j};	
	
}
\foreach \i/\j in{1/\(x+1\),2/\(-\),3/\(0\),4/\(+\),5/\(+\),6/\(+\),7/\(+\),8/\(+\)}{
	\node[anchor=center] at (S-2-\i.center){\j};	
	
}
\foreach \i/\j in{1/\(x^2-x-2\),2/\(+\),3/\(0\),4/\(-\),5/\(-\),6/\(-\),7/\(0\),8/\(+\)}{
	\node[anchor=center] at (S-3-\i.center){\j};	
	
}
\foreach \i/\j in{1/\(x\),2/\(-\),3/\(-\),4/\(-\),5/\(0\),6/\(+\),7/\(+\),8/\(+\)}{
	\node[anchor=center] at (S-4-\i.center){\j};	
	
}
\foreach \i/\j in{1/\(\frac{p(x)}{x}\),2/\(+\),3/\(0\),4/\(+\),6/\(-\),7/\(0\),8/\(+\)}{
	\node[anchor=center] at (S-5-\i.center){\j};	
	
}
\draw[top color=red, fill opacity=.2, decorate,decoration={brace,mirror,amplitude=1.5mm}](S-5-5.south west) to node[midway,fill opacity=1,below]{Decrescente} (S-5-7.south east);
\draw[top color=red, fill opacity=.2, decorate,decoration={brace,mirror,amplitude=1.5mm}](S-5-3.south west) to node[midway,fill opacity=1,below]{Zero} (S-5-3.south east);
%\fill[pattern=north west lines] (S-2-5.north west) rectangle (S-5-5.south east);
%\draw[->,>=stealth,shorten <=4mm,shorten >=4mm] (S-5-2.south west)--(S-5-2.north east);
%\draw[->,>=stealth,shorten <=4mm,shorten >=4mm] (S-5-4.north west)--(S-5-4.south east);
%\draw[->,>=stealth,shorten <=4mm,shorten >=4mm] (S-5-6.north west)--(S-5-6.south east);
%\draw[->,>=stealth,shorten <=4mm,shorten >=4mm] (S-5-8.south west)--(S-5-8.north east);
%\node[anchor=north] at (S-5-3.north){máx.};
%\node[anchor=south] at (S-5-3.south){\(-4\)};
%\node[anchor=north] at (S-5-7.north){min.};
%\node[anchor=south] at (S-5-7.south){\(4\)};
\fill[pattern=north west lines] (S-5-5.north west) rectangle (S-5-5.south east);
\end{tikzpicture}
\[C.S=\{-1\} \cup ]0,2]\]


\section*{Exercício 2}\textbf{Resolva, em $\mathbb{R}$, a inequação $\log(x - 4) - \log(10 - x) \geq 0$.}

\section*{Exercício 3}\textbf{Caracterize a função inversa da função $\it{g}$ definida por $g(x) = \log(2x + 5) + 1$.}

\section*{Exercício 4}\textbf{Considere a função real, de variável real, definida por $f(x) = 2 - e^x$ .}

\subsection*{a)}\textbf{Calcule as coordenadas do ponto de interseção do gráfico da função $\it{f}$ com a reta de equação $y = -5$.}

\subsection*{b)}\textbf{Determine o contradomínio da função $\it{f}$.}
\subsection*{c)}\textbf{Mostre que a reta tangente ao gráfico de $\it{f}$ no ponto de abcissa $0$ tem declive $-1$.}
\end{document}
